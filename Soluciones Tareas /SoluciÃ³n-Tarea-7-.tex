
\documentclass{oxmathproblems}
\usepackage{blindtext}
\usepackage{hyperref}
\usepackage{geometry}

\course{ITAM - Estadística 1}
\oxfordterm{Respuestas Assignment 07}
\sheetnumber{1}
\sheettitle{}

\begin{document}

\begin{questions}

\miquestion \textbf {Distribución Uniforme continua.} 
\begin{itemize}
\item  a) ¿El tiempo de reacción es una variable aleatoria o continua? ¿Cuál es la distribución de esta variable aleatoria?

\text{Sabemos que: }

$$$$ 
\[   
f(x) = 
     \begin{cases}
        \frac{1}{b-a} & (a \leq x \leq b)\\
       0 & \text{en otro caso} \\
     \end{cases}
\]
$$$$

\text{Entonces:  }

$[5,9] = [a,b]$

$\Rightarrow a = 5$  $b = 9$

$\Rightarrow$ 

$$$$ 
\[   
f(x) = 
     \begin{cases}
        \frac{1}{4} & (5 \leq x \leq 9)\\
       0 & \text{en otro caso} \\
     \end{cases}
\]
$$$$


\item  b)Determina la probabilidad de que el tiempo de reacción de una persona sea menor a 7 segundos, si se sabe que ya han pasado más de 6 segundos. 

$ P(x < 7 \mid x > 6) $ 
= 
$ \frac{P(x<7,x>6)}{P(x>6)}$

$\Rightarrow$  
\begin{equation}
 \int _{5}^{7}  \frac{1}{4}  \cdot dx
\end{equation} 

= 
\begin{equation}
 \frac{1}{4} \int _{5}^{7} \cdot dx
\end{equation} 
= 
$ \frac{1}{4}[x]_6^7$  = $\frac{1}{4}*(7-6) = \frac{1}{4} $ 

\text {Ahora: }
\begin{equation}
 \int _{6}^{9}  \frac{1}{4}  \cdot dx
\end{equation} 

= 
\begin{equation}
 \frac{1}{4} \int _{6}^{9} \cdot dx
\end{equation} 
= 
$ \frac{1}{4}[x]_6^9$  = $\frac{1}{4}*(9-6) = \frac{3}{4} $ 

$\Rightarrow$

$ P(x < 7 \mid x > 6) $ = $\frac{1}{3}$


\item  c) Valor esperado y desviación estándar del tiempo de reacción

$E(x) = \frac{a+b}{2} = \frac{5+9}{2} = 7$

\text {Ahora para la varianza }

$\sigma_x^2 = \frac{(b-a)^2}{12} = \frac{(9-5)^2}{12} = 1.33$

\text {Entonces la desviación estándar es :  }

$\sqrt1.33$ = 1.1532 
\end{itemize}


\miquestion \textbf {Distribución Uniforme continua.} 

\begin{itemize}
\item  a) El valor de k 

 
$ fx_(x) = $
$$$$
\[   
     \begin{cases}
\frac{1}{b-a} & (5 \leq x \leq 9)\\
       0 & \text{en otro caso} \\
     \end{cases}
\]
$$$$
$\Rightarrow$
 
 K = $\frac{1}{40}$

\item  b) Obtener la función de distribución acumulada de X

$\frac{x-a}{b-a}$  \text{para :}  $ a \leq x \leq b$

$\Rightarrow$

$ F_x(x) = \frac{x+20}{20-(-20)}  = \frac{x+20}{40)} $

\item  c) ¿Cuál es la probabilidad de que en un mes se tengan dos semanas con valores positivos de x?

$ P(x \geq 2)$ = $ 1/20 $
\end{itemize}

\miquestion
\begin{itemize}
\item a) La probabilidad de que la concentración de monóxido exceda las 6 partes por millón

$P(x>6)$
= 
\begin{equation}
 \frac{1}{2.5} \int_6^\infty {e^{\frac{-x}{2.5}}} \cdot dx
\end{equation}  
= 0.0907



\item b) En un momento dado se cuenta con la información de que w está entre 3 y 9. ¿Cuál es la probabilidad de que w sea mayor a 6?

$P(3<x>9)$
= 
\begin{equation}
 \frac{1}{2.5} \int_6^\infty {e^{\frac{-x}{2.5}}} \cdot dx
\end{equation}  
= 0.5
\end{itemize}


\end {questions}

\end{document}