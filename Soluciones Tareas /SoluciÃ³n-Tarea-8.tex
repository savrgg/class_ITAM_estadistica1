
\documentclass{oxmathproblems}
\usepackage{blindtext}
\usepackage{hyperref}
\usepackage{geometry}

\course{ITAM - Estadística 1}
\oxfordterm{ Respuestas Assignment 08}
\sheetnumber{1}
\sheettitle{}

\begin{document}

\begin{questions}

\miquestion \textbf {Variable aleatorias y distribuciones conjuntas.}

$P(X=2,Y=0) $ = 0.25



\miquestion\textbf{Distribución aleatoria conjunta}
\begin{itemize}
\item  a) La función de probabilidad conjunta de (X, Y)

\begin{center}
\begin{tabular}{ |c|c|c|c|c|c|c| } 
\hline
Y/X & 2 & 4 & 6 & 8 & 10 & 12 \\
\hline
1 & 1/6 & 0 & 1/6 & 0 & 1/6 & 0 \\
2 & 0 & 1/6 & 0 & 1/6 & 0 & 1/6 \\
\hline
\end{tabular}
\end{center}

\item  b) La probabilidad de Y sea par si X es menor o igual a 8. 


$P(Y = 2 \mid X \leq 8)$ =  $\frac{P(X \leq 8,Y = 2)}{P(X \leq 8)}$
= 1/2
\end{itemize}


\miquestion\textbf{Distribución aleatoria conjunta}

\text{Sabemos que: }
$ x,y \sim Norm(\begin{pmatrix} \mu_x \\ \mu_y \end{pmatrix} $ $\begin{pmatrix} \sigma_x^2  & \sigma_{xy}\\ \sigma_{xy} & \sigma_{y}^2\end{pmatrix} $) 

 \text{Entonces:  }

\begin{itemize}
\item  a) Coeficiente de correlación: 

 $ \rho_{xy} = \frac{\sigma_{xy}}{\sigma_x * \sigma_y} $ 

\text{x e y son bienes complementarios}


\item  b) 

\text{ w = x-y}
\text{entonces: }

$ P(x>y) = P (x-y >0) = P (z > \frac{5}{\sqrt11}) $ 

= 
$ P(z > 1.5076) = 0.0655 $

\end{itemize}

\miquestion \textbf {Aproximación de una Binomial a una normal}
\begin{itemize}
\item  a) 

$ P (x \geq 20 ) = 0.5517$ 

\item  b)  $ E(x) = n * p = 80 *0.75 = 60 $ 
\item  c)
$ P(10 \leq x \leq 30 ) = 0.9932 $ 
\end{itemize}


\end{questions}

\textbf{Bibliografía}
Mendenhall, W. (2006). Introducción a la probabilidad y Estadística (Vol. 13). Cengage Learning.

\end{document}