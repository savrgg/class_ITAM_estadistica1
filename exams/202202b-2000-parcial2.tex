%%% Template originaly created by Karol Kozioł (mail@karol-koziol.net) and modified for ShareLaTeX use

\documentclass[addpoints]{exam}

\usepackage[T1]{fontenc}
\usepackage[utf8]{inputenc}
\usepackage{graphicx}
\usepackage{xcolor}

\renewcommand\familydefault{\sfdefault}
\renewcommand{\theenumi}{\Alph{enumi}}
\usepackage{tgheros}

\usepackage{amsmath}
\usepackage{amssymb,amsthm,textcomp}
\usepackage{enumerate}
\usepackage{multicol}
\usepackage{tikz}
\usepackage[spanish, es-nodecimaldot]{babel}
\usepackage{enumitem}


\usepackage{geometry}
\geometry{left=25mm,right=25mm,bindingoffset=0mm, top=20mm,bottom=20mm}


\linespread{1.3}

\newcommand{\linia}{\rule{\linewidth}{0.5pt}}

% custom theorems if needed
\newtheoremstyle{mytheor}
{1ex}{1ex}{\normalfont}{0pt}{\scshape}{.}{1ex}
{{\thmname{#1 }}{\thmnumber{#2}}{\thmnote{ (#3)}}}
  
  \theoremstyle{mytheor}
  \newtheorem{defi}{Definition}
  
  % my own titles
  \makeatletter
  \renewcommand{\maketitle}{
    \begin{center}
    \vspace{2ex}
    {\huge \textsc{\@title}}
    \vspace{1ex}
    \\
    \linia\\
    \@author \hfill \@date
    \vspace{4ex}
    \end{center}
  }
  \makeatother
  %%%
  
  % custom footers and headers
  %\usepackage{fancyhdr}
  %\pagestyle{fancy}
  \lfoot{Parcial \textnumero{} 2}
  \cfoot{}
  \rfoot{Page \thepage}
  %\renewcommand{\headrulewidth}{0pt}
  %\renewcommand{\footrulewidth}{0pt}
  
  % code listing settings
  \usepackage{listings}
  \lstset{
    language=Python,
    basicstyle=\ttfamily\small,
    aboveskip={1.0\baselineskip},
    belowskip={1.0\baselineskip},
    columns=fixed,
    extendedchars=true,
    breaklines=true,
    tabsize=4,
    prebreak=\raisebox{0ex}[0ex][0ex]{\ensuremath{\hookleftarrow}},
    frame=lines,
    showtabs=false,
    showspaces=false,
    showstringspaces=false,
    keywordstyle=\color[rgb]{0.627,0.126,0.941},
    commentstyle=\color[rgb]{0.133,0.545,0.133},
    stringstyle=\color[rgb]{01,0,0},
    numbers=left,
    numberstyle=\small,
    stepnumber=1,
    numbersep=10pt,
    captionpos=t,
    escapeinside={\%*}{*)}
  }
  
  %%%----------%%%----------%%%----------%%%----------%%%
  
  \begin{document}
  
  \title{Parcial 2 - Estadística I}
  
  \author{ITAM, Otoño 2022}
  
  \date{14/11/2022}
  
  \maketitle
  
  \section*{Instrucciones}
  El examen se debe contestar individualmente y entregarse a más tardar a las 19:45. El examen cuenta con 6 preguntas a desarrollar. Se debe cuidar la formalidad al escribir los resultados, ya que es parte de la calificación del problema. En caso de no tener el desarrollo de la pregunta, o bien se llegué a la respuesta sin una justificación se podrá anular la respuesta. Cualquier práctica fraudulenta será sancionada de acuerdo al reglamento del departamento. \textbf{Trabajar con 4 cifras decimales}
  
\vspace{10pt}

  \section*{Seccion A: }
  
  \begin{questions}

\question \textbf{(20 pts)}
Se tiene una variable aleatoria X con media $\mu$ y varianza $\sigma^2$. Se define una transformación como: 

$$R(X) = \frac{(X-\mu)^2}{\sigma^2}$$

\begin{enumerate}[label=\Alph*)]
\item (6pts) Encuentra el valor esperado de la variable aleatoria R.
\end{enumerate}

Por otra parte se define Y como una variable aleatoria con media $\lambda$ y varianza $\gamma^2$. Se define una transformación como:

$$P(Y) = Y + \frac{2Y-\lambda}{\gamma}$$

\begin{enumerate}[label=\Alph*)]
\item (6pts) Encuentre el valor esperado de la variable P
\item (6pts) Encuentre la varianza de la variable aleatoria P
\end{enumerate}


\question \textbf{(20 pts)}
El consumo mensual de kilos de croquetas de Colmillo sigue la siguiente función de distribución:

\[   
f(x) = 
     \begin{cases}
       Kxe^{-\frac{x}{5}} & x \geq 0\\
       0 & \text{en otro caso} \\
     \end{cases}
\]
$$$$

\begin{enumerate}[label=\Alph*)]
\item (6pts) Determine el valor de k para que f(x) sea una función de distribución
\item (6pts) Obtenga la función de probabilidad acumulada. Compruebe que si X = límite inferior y Y = límite superior, entonces F(X) = 0 y F(Y) = 1
\item (3pts) Si para un mes particular solo se tienen 25 kilos de croqueta, ¿Cuál es la probabilidad que las croquetas sean insuficientes para colmillo?
\end{enumerate}

\question \textbf{(15 pts)}
El monto total de la cuenta en cientos de pesos (x) de un restaurante es una variable alatoria que sigue la siguiente función de densidad:


\[   
f(x) = 
     \begin{cases}
       a+bx & 0 \leq x \leq 10\\
       bx & 10 < x \leq 20\\
       0 & \text{en otro caso} \\
     \end{cases}
\]

Adicional, Se sabe que $P(0 \geq x \geq 10) = 0.5$

\begin{enumerate}[label=\Alph*)]
\item (5pts) Calcule los valores de a y b de tal manera que $f(x)$ sea una función de distribución (Nota, debido a que trabajamos con 4 decimales, puede ser que no integre exactamente 1, pero debe tener un rango de error de $\pm 0.01$)
\item (5pts) La utilidad neta (UN) para el restaurante está en función del monto de la cuenta y sigue la siguiente distribución: 
\end{enumerate}
$$ UN(X) = \frac{X}{2}-c $$
 
Donde $c$ es una constante dada. Obtenga el valor esperado, la moda y la varianza de la utilidad neta. (Usando los resultados de inciso a)

\question \textbf{(15 pts)}
El periodo de funcionamiento de un iphone hasta su primera falla (en años) se puede modelar por medio de la siguiente función de distribución acumulada:

\[   
F(y) = 
     \begin{cases}
       0 & y<0\\
      1-e^{-y^2} & y \geq 0 \\
     \end{cases}
\]

Es decir la probabilidad de que falle en 0 años o menos es F(0) = 1-exp(0) = 0, la probabilidad que falle en 3 años o menos es F(3) = 1-exp(-9)

\begin{enumerate}[label=\Alph*)]
\item (2pts) Comprueba que F(y) cumple con las propiedades de función de distribución acumulada
\item (2pts) Calcule la probabilidad que el iphone no falle entre 1.5 y 3.5 años
\item (4pts) Calcule la función de densidad correspondiente a la función de distribución acumulada
\item (2pts) Compruebe que f(y) cumple con las propiedades de función de distribución (En caso de no poder integrar f(x), use como argumento la función F(x))
\end{enumerate}


\end{questions} 



\end{document}
  

