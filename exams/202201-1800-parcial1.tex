%%% Template originaly created by Karol Kozioł (mail@karol-koziol.net) and modified for ShareLaTeX use

\documentclass[addpoints]{exam}

\usepackage[T1]{fontenc}
\usepackage[utf8]{inputenc}
\usepackage{graphicx}
\usepackage{xcolor}

\renewcommand\familydefault{\sfdefault}
\renewcommand{\theenumi}{\Alph{enumi}}
\usepackage{tgheros}

\usepackage{amsmath}
\usepackage{amssymb,amsthm,textcomp}
\usepackage{enumerate}
\usepackage{multicol}
\usepackage{tikz}
\usepackage[spanish, es-nodecimaldot]{babel}
\usepackage{enumitem}


\usepackage{geometry}
\geometry{left=25mm,right=25mm,bindingoffset=0mm, top=20mm,bottom=20mm}


\linespread{1.3}

\newcommand{\linia}{\rule{\linewidth}{0.5pt}}

% custom theorems if needed
\newtheoremstyle{mytheor}
{1ex}{1ex}{\normalfont}{0pt}{\scshape}{.}{1ex}
{{\thmname{#1 }}{\thmnumber{#2}}{\thmnote{ (#3)}}}
  
  \theoremstyle{mytheor}
  \newtheorem{defi}{Definition}
  
  % my own titles
  \makeatletter
  \renewcommand{\maketitle}{
    \begin{center}
    \vspace{2ex}
    {\huge \textsc{\@title}}
    \vspace{1ex}
    \\
    \linia\\
    \@author \hfill \@date
    \vspace{4ex}
    \end{center}
  }
  \makeatother
  %%%
  
  % custom footers and headers
  %\usepackage{fancyhdr}
  %\pagestyle{fancy}
  \lfoot{Parcial \textnumero{} 2}
  \cfoot{}
  \rfoot{Page \thepage}
  %\renewcommand{\headrulewidth}{0pt}
  %\renewcommand{\footrulewidth}{0pt}
  
  % code listing settings
  \usepackage{listings}
  \lstset{
    language=Python,
    basicstyle=\ttfamily\small,
    aboveskip={1.0\baselineskip},
    belowskip={1.0\baselineskip},
    columns=fixed,
    extendedchars=true,
    breaklines=true,
    tabsize=4,
    prebreak=\raisebox{0ex}[0ex][0ex]{\ensuremath{\hookleftarrow}},
    frame=lines,
    showtabs=false,
    showspaces=false,
    showstringspaces=false,
    keywordstyle=\color[rgb]{0.627,0.126,0.941},
    commentstyle=\color[rgb]{0.133,0.545,0.133},
    stringstyle=\color[rgb]{01,0,0},
    numbers=left,
    numberstyle=\small,
    stepnumber=1,
    numbersep=10pt,
    captionpos=t,
    escapeinside={\%*}{*)}
  }
  %%%----------%%%----------%%%----------%%%----------%%%
  
  %%%----------%%%----------%%%----------%%%----------%%%
  
  \begin{document}
  
  \title{Parcial 1 - Estadística I}
  
  \author{ITAM, Primavera 2022}
  
  \date{30/03/2022}
  
  \maketitle
  
  \section*{Instrucciones}
  
El examen tiene una duración de 1:40 horas y comienza a las 18:00 hrs. Se debe cuidar la formalidad al escribir los resultados, ya que es parte de la calificación del problema. En caso de no tener el desarrollo de la pregunta, o bien se llegué a la respuesta sin una justificación se podrá anular la respuesta. \textbf{Cualquier práctica fraudulenta será sancionada de acuerdo al reglamento.} 

\vspace{10pt}
  
  \section*{Seccion B: Preguntas a desarrollar (85 pts)}
  
  \begin{questions} 
  
  \question (15 pts) 
  Determine el \textbf{tamaño} del espacio muestral (cuantos elementos tiene, sin enumerarlos) para cada uno de los experimentos:
  \begin{enumerate}
  \item (5 pts) Se lanza 5 veces un dado de seis caras
  \item (5 pts) Una urna contiene cinco bolas rojas, seis blancas, una azul y tres amarillas. Una segunda contiene cinco pelotas blancas y tres azules. Se selecciona una bola de la primer urna y dos bolas de la segunda urna sin reemplazo. 
  \item (5 pts) Las urnas del ejercicio anterior son mezcladas en una urna y se extraen dos de ellas con reemplazo
  \end{enumerate}
  

\question (10 pts)
Sean A y B dos eventos tales que P(A) = 0.3 y P(B) = 0.2
\begin{enumerate}[label=\Alph*)]
\item (5 pts) Encuentre $P(A \cap B)$, $P(A \cup B)$, $P(A | B)$ si el evento A es mutuamente excluyente al evento B
\item (5 pts) Encuentre $P(A \cap B)$, $P(A \cup B)$, $P(A | B)$ si el evento A es independiente al evento B
\end{enumerate}

  \question (15 pts)
Se lanza un dado justo tal que la probabilidad de cada cara es 1/6. Si aparece alguno de los números 1,3 o 5 se extrae aleatoriamente una canica de una urna que contiene dos canicas rojas, tres verdes y una azul. Si aparecen alguno de 2 o 4, se extrae una canica de otra urna la cual contiene dos rojas, dos verdes y una azul. En otro caso, se extrae una canica de otra urna que contiene 1 roja.

 \begin{enumerate}[label=\Alph*)]
\item (4 pts) Construya el espacio muestral del experimento 
\item (4 pts) Calcule las probabilidades del cada elemento del espacio muestral
\item (7 pts) Calcule la probabilidad de extraer:
\subitem - Una canica roja 
\subitem - Una canica azul
\subitem - Una canica verde
\end{enumerate}
 
\question (10 pts)
Se ha hecho un estudio entre el consumo de refresco diario y tener obesidad. En un centro médico, de todos los pacientes que consumen refresco diario, el 90\% tiene obesidad, mientras que únicamente el 5\% de los que no consumen refresco diario la padecen. Si la proporción de personas que consumen refresco diario es de 15\%:
\begin{enumerate}[label=\Alph*)]
\item (5 pts) ¿Cuál es la probabilidad de que un paciente seleccionado al azar tenga obesidad?
\item (5 pts) ¿Cuál es la probabilidad de que un paciente con obesidad seleccionado al azar consuma refresco diario? 
\end{enumerate}

  \question (10 pts) Se lanzan 3 monedas que no son honestas, es decir sea $P(A)$ la probabilidad de aguila, la moneda A tiene una $P(A) = 2/3$, la moneda B una $P(A) = 1/2$ y la moneda C una $P(A) = 1/3$.
  
  \begin{enumerate}[label=\Alph*)]
  \item (3 pts) Escriba el espacio muestral del experimento aleatorio
  \item (3 pts) Obtenga la probabilidad de cada resultado del espacio muestral
  \item (2 pts) ¿Cuál es la probabilidad de observar al menos dos \textbf{soles}?
  \item (2 pts) ¿Cuál es la probabilidad que el último lanzamiento sea águila?
  \end{enumerate}
  
\question (15 pts) Se lanza un par de dados. Si se sabe que en un dado salió el número 3. Cuál es la probabilidad de que:
\begin{enumerate}[label=\Alph*)]
\item (5 pts) El otro número sea un 4
\item (5 pts) La suma de los números que salieron en los dados sea menor a 4
\item (5 pts) La suma de los números que salieron en los dados sea mayor a 7
\end{enumerate}

\question (10 pts) En una cafetería solo venden café colombiano y café mexicano, también existe la opción de con cafeína y sin cafeína (descafeinado). El 10\% de cafés que venden son mexicanos sin cafeina, 20\% son colombianos sin cafeina, 30\% mexicanos con cafeina y el resto colombianos con cafeina. Si se ordena un cafe al azar, encuentre la probabilidad de que:
\begin{enumerate}[label=\Alph*)]
\item (2 pts) El café sea sin cafeína
\item (2 pts) El café sea colombiano
\item (2 pts) El café sea mexicano o descafeinado
\item (2 pts) El café sea mexicano si se sabe que es descafeinado
\item (2 pts) El café sea descafeinado si se sabe que es mexicano

\end{enumerate}
\end{questions}


  \section*{Seccion A: Opción múltiple y V/F(15 pts)}
  
  \begin{questions}
  
  \question (5 pts) CONCEPTOS DE ESTADÍSTICA: Se realizará un estudio donde los elementos de interés son las agencias que venden automóviles. Identifique si la variable de interés es cualitativa o cuantitativa, discreta o continua y la escala de medición correspondiente. 
  
  \begin{enumerate}
  \item (1 pt) El número de automóviles por agencia
  \item (1 pt) Número de visitas a la agencia por mes
  \item (1 pt) Duración de las visitas más largas por mes
  \item (1 pt) Costo mensual de la renta del local de la agencia
  \item (1 pt) Temperatura en ºC promedio dentro de cada agencia
  \end{enumerate}
  
  \question (10 pts) Determine si las siguientes afirmaciones son verdaderas o falsas. Justifique
  
  \begin{enumerate}
  \item (1 pt) La media de un conjunto de datos no considera el valor máximo y minimo de los datos para su cálculo
  \item (1 pt) La moda puede tener más de tres valores 
  \item (1 pt) Es imposible que la moda, la mediana y la media sean iguales
  \item (1 pt) El coeficiente de variación es considerado un parámetro de centralidad
  \item (1 pt) El IQR es considerado un parámetro de dispersión
  \item (1 pt) El primer cuartil es menor o igual al mínimo de los datos
  \item (1 pt) La mediana debe ser igual al menos a uno de los valores observados
  \item (1 pt) La moda debe ser igual al menos a uno de los valores observados
  \item (1 pt) La media debe ser igual al menos a uno de los valores observados
  \item (1 pt) El coeficiente de correlación toma valores entre [0,1]
  \end{enumerate}

\end{questions}

\end{document}
  

