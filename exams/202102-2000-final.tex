%%% Template originaly created by Karol Kozioł (mail@karol-koziol.net) and modified for ShareLaTeX use

\documentclass[addpoints]{exam}

\usepackage[T1]{fontenc}
\usepackage[utf8]{inputenc}
\usepackage{graphicx}
\usepackage{xcolor}

\renewcommand\familydefault{\sfdefault}
\usepackage{tgheros}

\usepackage{amsmath}
\usepackage{amssymb,amsthm,textcomp}
\usepackage{enumerate}
\usepackage{multicol}
\usepackage{tikz}
\usepackage[spanish]{babel}

\usepackage{geometry}
\geometry{left=25mm,right=25mm,bindingoffset=0mm, top=20mm,bottom=20mm}


\linespread{1.3}

\newcommand{\linia}{\rule{\linewidth}{0.5pt}}

% custom theorems if needed
\newtheoremstyle{mytheor}
{1ex}{1ex}{\normalfont}{0pt}{\scshape}{.}{1ex}
{{\thmname{#1 }}{\thmnumber{#2}}{\thmnote{ (#3)}}}
  
  \theoremstyle{mytheor}
  \newtheorem{defi}{Definition}
  
  % my own titles
  \makeatletter
  \renewcommand{\maketitle}{
    \begin{center}
    \vspace{2ex}
    {\huge \textsc{\@title}}
    \vspace{1ex}
    \\
    \linia\\
    \@author \hfill \@date
    \vspace{4ex}
    \end{center}
  }
  \makeatother
  %%%
  
  % custom footers and headers
  %\usepackage{fancyhdr}
  %\pagestyle{fancy}
  \lfoot{Assignment \textnumero{} 5}
  \cfoot{}
  \rfoot{Page \thepage}
  %\renewcommand{\headrulewidth}{0pt}
  %\renewcommand{\footrulewidth}{0pt}
  
  % code listing settings
  \usepackage{listings}
  \lstset{
    language=Python,
    basicstyle=\ttfamily\small,
    aboveskip={1.0\baselineskip},
    belowskip={1.0\baselineskip},
    columns=fixed,
    extendedchars=true,
    breaklines=true,
    tabsize=4,
    prebreak=\raisebox{0ex}[0ex][0ex]{\ensuremath{\hookleftarrow}},
    frame=lines,
    showtabs=false,
    showspaces=false,
    showstringspaces=false,
    keywordstyle=\color[rgb]{0.627,0.126,0.941},
    commentstyle=\color[rgb]{0.133,0.545,0.133},
    stringstyle=\color[rgb]{01,0,0},
    numbers=left,
    numberstyle=\small,
    stepnumber=1,
    numbersep=10pt,
    captionpos=t,
    escapeinside={\%*}{*)}
  }
  
  %%%----------%%%----------%%%----------%%%----------%%%
  
  \begin{document}
  
  \title{Propuesta Final - Estadística I}
  
  \author{ITAM, Primavera 2022}
  
  \date{21/05/2022}
  
  \maketitle
  
  \section*{Instrucciones}
  
\vspace{10pt}
  

  
  \section*{Poisson}
  
  \begin{questions}
  
  \question (5 pts) El Centro de Aprendizaje Redacción y Lenguas estudió el número de errores ortográficos que cometen los estudiantes del ITAM al escribir un reporte de una cuartilla (X). Encontraron que la media de errores por cuartilla es de 3. 
  \begin{enumerate}
  \item ¿Qué distribución se puede emplear para modelar los errores (X)?. Determine su media y su varianza\\
  \textcolor{red}{$X \sim Pois(3), \\ E(X) = 3, \\ V(X) = 3$}
  \item Determine ¿Cuál es la probabilidad que un estudiante cometa al menos 4 errores en la cuartilla? ¿Cuál es la probabilidad que un estudiante cometa exactamente 4 errores en la cuartilla? \\
  \textcolor{red}{$P(X >= 4) =  1-P(X <= 3) = 1- 0.647 = 0.353$} \\
  \textcolor{red}{$P(X = 4) =  P(X <= 4)-P(X <= 3) = 0.815 - 0.647 = 0.168$}
  \item La calificación de la cuartilla se puede modelar como la función $C(X)=100-\frac{X^2}{6}$. ¿Cuál es la calificación esperada de esta cuartilla? \\
  \textcolor{red}{$E(X^2) = V(X)+E(X)^2 = 3 + 9 = 12$} \\
  \textcolor{red}{$E(C) = E(100-\frac{X^2}{6}) = 100-\frac{E(X^2)}{6} = 100 - \frac{12}{6} = 98$}

  \item Se ha revisado una proporción de la cuartilla y se sabe que hay al menos 2 errores. ¿Cuál es la probabilidad que se tengan al menos 3 errores en la cuartilla completa? \\
  \textcolor{red}{$P(X >= 3 | X >= 2 ) = \frac{P(X >= 3,X >= 2 )}{P(X >= 2)} = \frac{P(X >= 3)}{P(X>=2)} = \frac{1-P(X<=2)}{1-P(X<=1)} = \frac{0.577}{0.801} = 0.7203$}
  
  \end{enumerate}
  
    \question (5 pts) El número de llamadas que recibe el conmutador del ITAM por hora se puede modelar como una variable Poisson (X) con media de 15. Debido a condiciones del conmutador se conoce que a lo máximo atienda 20 llamadas por hora y deja las demás sin atender. 

  \begin{enumerate}
  \item ¿Cuál es la probabilidad que se dejen llamadas sin atender? \\
  \textcolor{red}{$P(X > 20) = 1- P(X<= 20) = 1-0.917 = 0.083$}
  \item El ITAM incurre en un costo por llamada (tanto por las atendidas como por las no atendidas). El costo se puede modelar como: $C(X) = 100+10X$. Encuentre el Valor Esperado y la varianza del costo. \\
  \textcolor{red}{$E(C) = E(100+10X) = 100+10E(X) = 100+10(15) = 250$} \\
  \textcolor{red}{$V(C) = V(100+10X) = 100V(X) = 100(15) = 150$}
  
  \item Se sabe que se han recibido al menos 10 llamadas en una hora, ¿Cual es la probilidad de que se tengan más de 20 llamadas en esa hora? \\
  \textcolor{red}{$P(X>20|X>=10) = \frac{P(X>20, X>=10)}{P(X>=10)} = \frac{P(X>20)}{P(X>=10)} = \frac{1 - P(X<=20)}{1-P(X<=9)} = \frac{1-0.917}{1-0.07} = \frac{0.083}{0.93} = 0.0892$}

  \end{enumerate}
  
  \section*{Geométrica}
  
  \question Un equipo de auditoria encuentra que 9 de cada 10 auditorias realizadas encuentra fallas en los procesos auditados. Sea X la variable aleatoria que denota el número de auditorias realizadas hasta encontrar la primer falla en proceso. Si el equipo realiza una serie de auditorias:
   \begin{enumerate}
   \item ¿Cuantos auditorias se esperan realizar para encontrar la primer falla?
   Adicional, encuentre E(X(X-1)) \\
   \textcolor{red}{$E(X) = \frac{1}{0.9} = 1.1111$} \\
   \textcolor{red}{$E(X^2 -X) = V(X)+E(X)^2-E(X) = \frac{1-0.9}{0.9^2} + 1.1111^2 - 1.1111 = 0.2469$}
   
   \item ¿Cuál es la probabilidad que la tercer auditoria realizada sea la primera que encuentre fallas en los procesos?. \\
   \textcolor{red}{$P(X = 3) = (0.1)^2(0.9) = 0.009$}
   \item Se sabe que no encontraron falla en los primeros dos procesos. ¿Cuál es la probabilidad que la primer falla se encuenter en la cuarta auditoria ? \\
    \textcolor{red}{$P(X = 4 | X > 2) = \frac{P(X = 4, X > 2)}{P(X>2)} = \frac{P(X =4)}{P(X>2)} = \frac{P(X =4)}{1-P(X<=2)} = \frac{0.1^3 (0.9)}{0.01} = 0.09$}
   \end{enumerate}
  

\question Una empresa minera realizará una serie de exploraciones para encontrar plata. Se sabe que 1 de cada 5 exploraciones son exitosas. 
   \begin{enumerate}
   \item ¿Cuál es la probabilidad que la primera exploración sea la primera exitosa? \\
   \textcolor{red}{$P(X = 1) = 0.2^1 (0.8^0)$}
   \item ¿Cuál es la probabilidad que la primer exploración exitosa este entre la cuarta y sexta exploración (inclusive)? \\
   \textcolor{red}{$P( 4 <= X = 6) = 0.2^1 (0.8^3)+ 0.2^1 (0.8^4) + 0.2^1 (0.8^5) = 0.249856$}
   \item ¿Cuántas exploraciones se esperan realizar para encontrar la primer exploración exitosa? \\
   \textcolor{red}{$E(X) = 1/0.20 = 5$}
   \item ¿Se tiene que el costo de las exploraciones sigue una función $C(X) = 100X+5X^2$ ¿Cuál es el costo esperado que se incurre para encontrar la primer exploración exitosa?\\
   \textcolor{red}{$E(X^2) = V(X)+E(X)^2 = \frac{1-0.2}{0.2^2} + 25 = 20 + 25 = 45$} \\
   \textcolor{red}{$E(C) = E(100X+5X^2) = 100E(X)+5E(X^2) = 100*5 + 5(45) = 725$}
   
   \end{enumerate}
  
  

  \question 3 estudiantes del ITAM, 3 de la Anáhuac y 2 de la Ibero se postulan a un doctorado en Estados unidos. De este grupo de 8 postulantes al doctorado solamente se escogerán 2. Se sabe que cada uno los 8 estudiantes tienen la misma probabilidad de ser seleccionados, por lo que sea X la variable que indica el número de estudiantes del ITAM seleccionados y Y el número de estudiantes de la Anáhuac seleccionados. 
  
     \begin{enumerate}
   \item Construya la tabla de probabilidad conjunta de X y Y \\
  
\begin{center}
\begin{tabular}{ c c c c }

 (X,Y) & 0 & 1 & 2\\ 
 ------- & ------- & ------- & -------\\
 0 | & 1/28 & 6/28 & 3/28\\  
 1 | & 6/28 & 9/28 & 0 \\    
 2 | & 3/28 & 0 & 0   
\end{tabular}
\end{center}

   \item Encuentre las distribuciones marginales de X y Y
   
    \textcolor{red}{$$P(X = 0) = 10/28, P(X = 1) = 15/28, P(X = 2) = 3/28 $$}
    \textcolor{red}{$$P(Y = 0) = 10/28, P(Y = 1) = 15/28, P(Y = 2) = 3/28 $$}
   
   \item ¿Calcule el valor esperado y varianza de X y de Y? \\
    \textcolor{red}{$$E(X) = 0.75, V(X) = 0.4018$$}
    \textcolor{red}{$$E(Y) = 0.75, V(Y) = 0.4018$$}
   \item Obtenga el coeficiente de correlación entre X y Y
    \textcolor{red}{$$E(XY) = 0.3214$$}
    \textcolor{red}{$$C(X,Y) = 0.3214-0.75*0.75 = -0.2411$$}
    \textcolor{red}{$$Cor(X,Y) = -0.2411/(sqrt(.4018)*sqrt(.4018)) = -0.6000$$}
   \item Si Z es el número de estudiantes que son de la Ibero, obtenga su valor esperado y la varianza.
    \textcolor{red}{$$P(Z = 0) = 15/28, P(Z = 1) = 12/28, P(Z = 2) = 1/28 $$}
    \textcolor{red}{$$E(Z) = E(2-X-Y) = 2-E(X)-E(Y) = 0.5$$}
    \textcolor{red}{$$V(Z) = 0.3214$$}
   \end{enumerate}
  
  
  
  \section*{Multivariada}
  
    \question Se lanzan 3 dados, sea X el número de veces que sale una cara par en el lanzamiento y Y en número de veces que sale una cara impar en el lanzamiento:
    
\begin{enumerate}
\item Obtenga la distribución conjunta de X y Y \\
\begin{center}
\begin{tabular}{ c c c c c}

 (X,Y) & 0 & 1 & 2 & 3\\ 
 ------- & ------- & ------- & ------- & -------\\
 0 | & 0 & 0 & 0 & 1/8\\  
 1 | & 0 & 0 & 3/8 & 0\\    
 2 | & 0 & 3/8 & 0 & 0\\
 3 | & 1/8 & 0 & 0 & 0  
\end{tabular}
\end{center}
   \item Encuentre las distribuciones marginales de X y Y
   
    \textcolor{red}{$$P(X = 0) = 1/8, P(X = 1) = 3/8, P(X = 2) = 3/8, P(X = 3) = 1/8 $$}
    \textcolor{red}{$$P(Y = 0) = 1/8, P(Y = 1) = 3/8, P(Y = 2) = 3/8, P(Y = 3) = 1/8 $$}
   
   \item ¿Calcule el valor esperado y varianza de X y de Y? \\
    \textcolor{red}{$$E(X) = 1.5, V(X) = 0.75$$}
    \textcolor{red}{$$E(Y) = 1.5, V(Y) = 0.75$$}
   
   \item Obtenga el coeficiente de correlación entre X y Y
    \textcolor{red}{$$E(XY) = 1.5$$}
    \textcolor{red}{$$C(X,Y) = 1.5-1.5*1.5 = -0.75$$}
    \textcolor{red}{$$Cor(X,Y) = -0.75/(sqrt(0.75)*sqrt(0.75)) = -1$$}
   
   \item ¿Qué puede concluir de los valores observados en los incisos anteriores?
   \end{enumerate}
  
   
\end{questions} 
\end{document}
  

