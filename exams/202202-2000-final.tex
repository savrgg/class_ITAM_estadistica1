%%% Template originaly created by Karol Kozioł (mail@karol-koziol.net) and modified for ShareLaTeX use

\documentclass[addpoints]{exam}

\usepackage[T1]{fontenc}
\usepackage[utf8]{inputenc}
\usepackage{graphicx}
\usepackage{xcolor}

\renewcommand\familydefault{\sfdefault}
\usepackage{tgheros}

\usepackage{amsmath}
\usepackage{amssymb,amsthm,textcomp}
\usepackage{enumerate}
\usepackage{multicol}
\usepackage{tikz}
\usepackage[spanish]{babel}

\usepackage{geometry}
\geometry{left=25mm,right=25mm,bindingoffset=0mm, top=20mm,bottom=20mm}


\linespread{1.3}

\newcommand{\linia}{\rule{\linewidth}{0.5pt}}

% custom theorems if needed
\newtheoremstyle{mytheor}
{1ex}{1ex}{\normalfont}{0pt}{\scshape}{.}{1ex}
{{\thmname{#1 }}{\thmnumber{#2}}{\thmnote{ (#3)}}}
  
  \theoremstyle{mytheor}
  \newtheorem{defi}{Definition}
  
  % my own titles
  \makeatletter
  \renewcommand{\maketitle}{
    \begin{center}
    \vspace{2ex}
    {\huge \textsc{\@title}}
    \vspace{1ex}
    \\
    \linia\\
    \@author \hfill \@date
    \vspace{4ex}
    \end{center}
  }
  \makeatother
  %%%
  
  % custom footers and headers
  %\usepackage{fancyhdr}
  %\pagestyle{fancy}
  \lfoot{Assignment \textnumero{} 5}
  \cfoot{}
  \rfoot{Page \thepage}
  %\renewcommand{\headrulewidth}{0pt}
  %\renewcommand{\footrulewidth}{0pt}
  
  % code listing settings
  \usepackage{listings}
  \lstset{
    language=Python,
    basicstyle=\ttfamily\small,
    aboveskip={1.0\baselineskip},
    belowskip={1.0\baselineskip},
    columns=fixed,
    extendedchars=true,
    breaklines=true,
    tabsize=4,
    prebreak=\raisebox{0ex}[0ex][0ex]{\ensuremath{\hookleftarrow}},
    frame=lines,
    showtabs=false,
    showspaces=false,
    showstringspaces=false,
    keywordstyle=\color[rgb]{0.627,0.126,0.941},
    commentstyle=\color[rgb]{0.133,0.545,0.133},
    stringstyle=\color[rgb]{01,0,0},
    numbers=left,
    numberstyle=\small,
    stepnumber=1,
    numbersep=10pt,
    captionpos=t,
    escapeinside={\%*}{*)}
  }
  
  %%%----------%%%----------%%%----------%%%----------%%%
  
  \begin{document}
  
  \title{Propuesta Final - Estadística I}
  
  \author{ITAM, Otoño 2022}
  
  \date{02/12/2022}
  
  \maketitle
  
  \section*{Instrucciones}
  
\vspace{10pt}
  
  
  \begin{questions}
  
  \section*{Bayes}
\question Se analiza el uso del Aeropuerto Internacional de la Ciudad de México (AICM) y se encuentra que el 60\% de las personas llega a través de vuelos comerciales low-cost, 30\% a través de vuelos comerciales que no son low-cost y el 10\% en vuelos privados. Adicionalmente se sabe que el 50\% de los que llegan por vuelos comerciales que no son low-cost viajan por negocios, el 80\% de los que llegan por medio de vuelo privado viajan por negocios y por ultimo el 10\% de los que viajan en low-cost viajan por negocios.  Si se selecciona una persona al azar en el aeropuerto:
\begin{enumerate}
\item ¿Cuál es la probabilidad que viaje por negocios?
\item ¿Cuál es la probabilidad que viaje por negocios en avión privado?
\item ¿Cuál es la probabilidad que llegue en avión privado, dado que la persona viaja por negocios?
\item ¿Cuál es la probabilidad que viaje por negocio dado que vuela en avión comercial?
\end{enumerate}

\question En un zoologico se busca analizar la cantidad de tortugas enfermas que se deberán atender en el año. Se sabe que este número es dependiente del alimento que se les proporcione. El 30\% de las tortugas se les alimenta con la marca A, el 50\% con la marca B y el 20\% restante con la marca C. Se sabe que dado que se alimenta con la marca A se enferman el 10\%, dado que se alimentan con marca B se enferman el 15\% y dado que se alimentan con la marca C se enferman el 20\%. Entonces:
\begin{enumerate}
\item ¿Cuál es la probabilidad que una tortuga se enferme si se desconoce la marca de comida que consume?
\item ¿Cuál es la probabilidad que dado que se observa una tortuga enferma, se haya alimentado con marca A?
\item ¿Cuál es la probabilidad que dado que se observa una tortuga enferma, se haya alimentado con marca A o B?
\item ¿Cuál es la probabilidad que dado que una tortuga esté sana dado que sabe que se alimento con el alimento A o B?
\end{enumerate}

  \section*{Geométrica}
  
  \question Al realizar una votación se sabe que en 1 de cada 10 casillas electorales se encuentran anomalías. Debido a este problema, se realiza una auditoria que busca encontrar una casilla anómala para examinarla a profundidad. Sea Z el número de casillas examinadas necesarias para encontrar la primer casilla anómala (Por ejemplo, si Z = 5, implica que la casilla 5 fue la primera anómala encontrada)
   \begin{enumerate}
   \item ¿Cuál es el número esperado de casillas examinadas para encontrar la primera anomalía? 
   
   \item Se tiene que el costo en miles de la auditoria sigue la siguiente función: $C(Z)=Z(Z-1)$. ¿Cuál será el costo esperado de la auditoria?
   
   \item ¿Cuál es la probabilidad que la tercer casilla sea la primer casilla anómala encontrada?
   
   \item Si ya se examinaron 2 casillas electorales y no se encontró ninguna anomalía, ¿Cuál es la probabilidad que la séptima casilla presente anomalías?

   \end{enumerate}
  
\section*{Binomial}
\question El final de economía del ITAM cuenta con 2 secciones: A y B, cada una con 5 preguntas, numeradas del 1 al 5. De la sección A los alumnos deben seleccionar solo una pregunta para resolverla y de la sección B solo una pregunta para resolverla (Suponga que escogen cada pregunta con la misma probabilidad).
\begin{enumerate}
   \item ¿Cuál es la probabilidad que un alumno al azar en la sección A seleccione una pregunta par y en la sección B seleccione una pregunta impar?
   \item Si 15 alumnos presentaron el final, ¿Cuál es la probabilidad que al menos 5 alumnos hayan elegido una pregunta par en la sección A y una pregunta impar en la sección B?
   \item ¿Cómo se modifica la probabilidad del inciso anterior si se sabe que al menos 2 alumnos seleccionaron una pregunta par en la sección A y una pregunta impar en la sección B?
\end{enumerate}

\section*{Poisson}
\question Un partido electoral desea conocer el número distritos electorales que ganarán en tres estados durante las elecciones (Nuevo León, Sonora y Tamaulipas). El número de distritos ganados en cada estado se pueden denotar con las variables aleatorias X, Y, Z respectivamente y se sabe que X ~ Pois(3), Y ~ Pois(2) y Z ~ Pois(1). El número de distritos electorales ganados en independiente entre estados. 

\begin{enumerate}
\item ¿Cuál es la probabilidad que ese partido electoral gane al menos 1 distritos electorales en cada estado?
\item ¿Cuál es la probabilidad que en al menos 1 estado ganen más de 2 distritos electorales?
\item ¿Cuál es la probabilidad que ganen menos de 2 distritos electorales en cada uno de los estados?
\item Si suman el número total de distritos ganados en cada estado (Nuevo León+ Sonora+Tamaulipas) ¿Cuál es el número esperado de distritos electorales ganado entre los tres estados?
\end{enumerate}

\section*{Exponencial}
\question El Departamento de Estadística decide que el examen final será realizado de manera oral. De experimentos previos se sabe que el tiempo que toma cada estudiante en realizar su examen se distribuye de manera exponencial con media de 10 minutos. Si se tienen solo 5 alumnos que presentarán el final (Suponga que el tiempo de resolución entre alumnos es independiente):

\begin{enumerate}
\item ¿Cuál es la probabilidad que ningún alumno tarde más de 10 minutos en resolver el examen oral?
\item ¿Cuál es la probabilidad que al menos uno de los alumnos tarde más de 20 minutos en responder el examen?
\item  ¿Cuál es el tiempo total esperado que tendrá que ocupar el departamento para realizar los 5 examenes si los realiza de manera secuencial (es decir, no los puede realizar en paralelo)?
\end{enumerate}
   
\end{questions} 
\end{document}
  

