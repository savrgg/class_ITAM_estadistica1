%%% Template originaly created by Karol Kozioł (mail@karol-koziol.net) and modified for ShareLaTeX use

\documentclass[addpoints]{exam}

\usepackage[T1]{fontenc}
\usepackage[utf8]{inputenc}
\usepackage{graphicx}
\usepackage{xcolor}

\renewcommand\familydefault{\sfdefault}
\usepackage{tgheros}

\usepackage{amsmath}
\usepackage{amssymb,amsthm,textcomp}
\usepackage{enumerate}
\usepackage{multicol}
\usepackage{tikz}
\usepackage[spanish]{babel}

\usepackage{geometry}
\geometry{left=25mm,right=25mm,bindingoffset=0mm, top=20mm,bottom=20mm}


\linespread{1.3}

\newcommand{\linia}{\rule{\linewidth}{0.5pt}}

% custom theorems if needed
\newtheoremstyle{mytheor}
{1ex}{1ex}{\normalfont}{0pt}{\scshape}{.}{1ex}
{{\thmname{#1 }}{\thmnumber{#2}}{\thmnote{ (#3)}}}
  
  \theoremstyle{mytheor}
  \newtheorem{defi}{Definition}
  
  % my own titles
  \makeatletter
  \renewcommand{\maketitle}{
    \begin{center}
    \vspace{2ex}
    {\huge \textsc{\@title}}
    \vspace{1ex}
    \\
    \linia\\
    \@author \hfill \@date
    \vspace{4ex}
    \end{center}
  }
  \makeatother
  %%%
  
  % custom footers and headers
  %\usepackage{fancyhdr}
  %\pagestyle{fancy}
  \lfoot{Assignment \textnumero{} 5}
  \cfoot{}
  \rfoot{Page \thepage}
  %\renewcommand{\headrulewidth}{0pt}
  %\renewcommand{\footrulewidth}{0pt}
  
  % code listing settings
  \usepackage{listings}
  \lstset{
    language=Python,
    basicstyle=\ttfamily\small,
    aboveskip={1.0\baselineskip},
    belowskip={1.0\baselineskip},
    columns=fixed,
    extendedchars=true,
    breaklines=true,
    tabsize=4,
    prebreak=\raisebox{0ex}[0ex][0ex]{\ensuremath{\hookleftarrow}},
    frame=lines,
    showtabs=false,
    showspaces=false,
    showstringspaces=false,
    keywordstyle=\color[rgb]{0.627,0.126,0.941},
    commentstyle=\color[rgb]{0.133,0.545,0.133},
    stringstyle=\color[rgb]{01,0,0},
    numbers=left,
    numberstyle=\small,
    stepnumber=1,
    numbersep=10pt,
    captionpos=t,
    escapeinside={\%*}{*)}
  }
  
  %%%----------%%%----------%%%----------%%%----------%%%
  
  \begin{document}
  
  \title{Parcial 2 - Estadística I}
  
  \author{ITAM, Otoño 2021}
  
  \date{04/11/2021}
  
  \maketitle
  
  \section*{Instrucciones}
\vspace{10pt}

  \section*{Seccion A: }
  
  \begin{questions}
  
\question (10 pts) En un zoológico deciden hacer una ampliación al área de elefantes por lo que se realiza una licitación donde se postulan 4 empresas. La primer empresa promete terminar la obra en 20 dias, la segunda empresa acabarlo en 25, la tercera empresa en 30 dias y la cuarta en 35 dias. Debido a cuestiones legales, todas las empresas tienen la misma probabilidad de ser elegidas. Considerando estos datos:

¿Cuál es el tiempo esperado para terminar la obra?
¿Cuál es su varianza? Interpreta

El costo de la obra está en función del tiempo, por lo que a menor tiempo de obra se incurren en mayores costos. En particular, se puede modelar la función del costo como:

C(T) = 10T**2+1000T+505

Calcule el valor esperado y la varianza del costo. Interpreta

\question (10 pts) 
Una marca de whisky llamada Juanito Caminante cuenta con distintas submarcas que se venden de acuerdo a la siguiente distribución de probabilidad f(x). Adicional, se presenta el ingreso por botella:

Submarca, probabilidad, ingreso por botella
Paliacate rojo c/2! 300
Paliacate negro c/3! 800
Paliacate dorado c/3! 1200
Paliacate verde c/4! 1800
Paliacate azul c/5! 5000

a) Obtenga el valor de c que hace que f(x) sea una función de distribución.

b) Obtenga el valor esperado del costo por botella y la varianza del costo por botella 

c) Si se venden 1000 botellas al dia, encuentre el ingreso esperado para un día de venta 

\question (10 pts) 
Un alumno de una prestigiosa universidad decide ir a una fiesta de Halloween. El número de refrescos que tomará en toda la fiesta sigue la siguiente distribución:

x p(x)
1 0.1
2 0.2
3 0.25
4 0.2
5 0.25

a) Encuentre el valor esperado, la moda, la mediana y la desviación estándar del número de refrescos

b) Grafique la función de probabilidad

c) Obtenga la función de probabilidad acumulada y grafíquela

\question (10 pts) 
El consumo diario de croquetas de Colmillo sigue la siguiente función de distribución:

Kxe**(-x/5)

a) Determine el valor de k para que f(x) sea una función de distribución
b) Obtenga la función de probabilidad acumulada
c) Si para un dia particular solo se tienen xx kilos de croqueta, ¿Cuál es la probabilidad que las croquetas sean insuficientes para colmillo?

\question (10 pts) 
El costo total de la cuenta de un restaurante sigue una variable alatoria que sigue la siguiente función de densidad:

f(x) = a+bx  para 0<x<2000

a) Calcule los valores de a y b tal que el valor esperado del costo total sea igual a 1500

b) La utilidad neta para el restaurante está en función del monto de la cuenta y sigue la siguiente distribución: 

UN = 1/2x+c

Obtenga el valor esperado, la moda y la varianza de la utilidad neta.

\question (10 pts) 
A partir de la siguiente tabla de probabilidad conjunta, calcule las siguientes probabilidades:
     x
y  0 1 2 
0
1
2
3

1) Obtenga la distribución marginal de x y de y
2) obtenga las siguientes probabilidades:
P(X<=1)
P(Y>1)
P(X<=1, y<=1)
P(Y-X>2)
P(X>0|Y=2)
3) Obtenga el coeficiente de correlación de X y Y
4) Obtenga E(Y-X) y Var(Y-X)

\question (10 pts) 
Se lanzan dos dados. Sea X el número de unos que salen y Y el número de seis que aparecen.
a) Obtenga la función de probabilidad conjunta de X y Y
b) ¿Son independientes X y Y? Justifique por medio de probabilidad
c) Calcule la covarianza y el coeficiente de correlación entre X y Y
d) Si Z es el número de dados con valor diferente a 1 y 6, obtenga el valor esperado y la varianza de Z

\question (10 pts) 
Se proporciona una función de distribución acumulada, 

Una de F(x)

\question (10 pts) 
una de integración por partes





\end{questions} 



\end{document}
  

