%%% Template originaly created by Karol Kozioł (mail@karol-koziol.net) and modified for ShareLaTeX use

\documentclass[addpoints]{exam}

\usepackage[T1]{fontenc}
\usepackage[utf8]{inputenc}
\usepackage{graphicx}
\usepackage{xcolor}

\renewcommand\familydefault{\sfdefault}
\renewcommand{\theenumi}{\Alph{enumi}}
\usepackage{tgheros}

\usepackage{amsmath}
\usepackage{amssymb,amsthm,textcomp}
\usepackage{enumerate}
\usepackage{multicol}
\usepackage{tikz}
\usepackage[spanish, es-nodecimaldot]{babel}
\usepackage{enumitem}


\usepackage{geometry}
\geometry{left=25mm,right=25mm,bindingoffset=0mm, top=20mm,bottom=20mm}


\linespread{1.3}

\newcommand{\linia}{\rule{\linewidth}{0.5pt}}

% custom theorems if needed
\newtheoremstyle{mytheor}
{1ex}{1ex}{\normalfont}{0pt}{\scshape}{.}{1ex}
{{\thmname{#1 }}{\thmnumber{#2}}{\thmnote{ (#3)}}}
  
  \theoremstyle{mytheor}
  \newtheorem{defi}{Definition}
  
  % my own titles
  \makeatletter
  \renewcommand{\maketitle}{
    \begin{center}
    \vspace{2ex}
    {\huge \textsc{\@title}}
    \vspace{1ex}
    \\
    \linia\\
    \@author \hfill \@date
    \vspace{4ex}
    \end{center}
  }
  \makeatother
  %%%
  
  % custom footers and headers
  %\usepackage{fancyhdr}
  %\pagestyle{fancy}
  \lfoot{Assignment \textnumero{} 5}
  \cfoot{}
  \rfoot{Page \thepage}
  %\renewcommand{\headrulewidth}{0pt}
  %\renewcommand{\footrulewidth}{0pt}
  
  % code listing settings
  \usepackage{listings}
  \lstset{
    language=Python,
    basicstyle=\ttfamily\small,
    aboveskip={1.0\baselineskip},
    belowskip={1.0\baselineskip},
    columns=fixed,
    extendedchars=true,
    breaklines=true,
    tabsize=4,
    prebreak=\raisebox{0ex}[0ex][0ex]{\ensuremath{\hookleftarrow}},
    frame=lines,
    showtabs=false,
    showspaces=false,
    showstringspaces=false,
    keywordstyle=\color[rgb]{0.627,0.126,0.941},
    commentstyle=\color[rgb]{0.133,0.545,0.133},
    stringstyle=\color[rgb]{01,0,0},
    numbers=left,
    numberstyle=\small,
    stepnumber=1,
    numbersep=10pt,
    captionpos=t,
    escapeinside={\%*}{*)}
  }
  
  %%%----------%%%----------%%%----------%%%----------%%%
  
  \begin{document}
  
  \title{Parcial 2 - Estadística I}
  
  \author{ITAM, Otoño 2021}
  
  \date{04/11/2021}
  
  \maketitle
  
  \section*{Instrucciones}
  El examen es para resolver en casa. Se debe contestar individualmente y entregarse a más tardar a las 23:59 del viernes 13 de octubre. La entrega será por canvas. El examen cuenta con 8 preguntas a desarrollar. Se debe cuidar la formalidad al escribir los resultados, ya que es parte de la calificación del problema. En caso de no tener el desarrollo de la pregunta, o bien se llegué a la respuesta sin una justificación se podrá anular la respuesta. Cualquier práctica fraudulenta será sancionada de acuerdo al reglamento del departamento. \textbf{Trabajar con 4 cifras decimales}
  
\vspace{10pt}

  \section*{Seccion A: }
  
  \begin{questions}
  
\question \textbf{(15 pts)} En un zoológico deciden hacer una ampliación al área de elefantes por lo que se realiza una licitación donde se postulan 4 empresas. La primer empresa promete terminar la obra en 20 dias, la segunda empresa acabarlo en 25, la tercera empresa en 30 dias y la cuarta en 35 dias. Debido a cuestiones legales, todas las empresas tienen la misma probabilidad de ser elegidas. Considerando estos datos:

\begin{enumerate}[label=\Alph*)]
\item (2pts) ¿Cuál es el tiempo esperado para terminar la obra?
\item (2pts) ¿Cuál es la varianza? 
\item (1pt) Interpreta el resultado
\end{enumerate}

El costo de la obra está en función del tiempo, por lo que a menor tiempo de obra se incurren en mayores costos. En particular, se puede modelar la función del costo como:


$$C(T) = 10T^2+1000T+505$$


\begin{enumerate}[label=\Alph*)]
\item (2pts) Calcule el valor esperado del costo
\item (7pts) Calcule la varianza del costo. 
\item (1pt) Interpreta el resultado
\end{enumerate}

Al no ser independiente en la covarianza, se debe calcular usando la fórmula vista en clase: $cov(X,Y) = E(XY) - E(X)E(Y)$

\question \textbf{(10 pts)}
Una marca de whisky llamada Juanito Caminante cuenta con distintas submarcas que se venden de acuerdo a la siguiente distribución de probabilidad f(x). Adicional, se presentan los ingresos por botella:

\begin{center}
\begin{tabular}{ |c|c|c|c| } 
\hline
Submarca & Probabilidad & Costo por botella (C) & Ingreso por botella (I)\\
 \hline
 Paliacate rojo & $\frac{k}{2!}$ & 200 & 300 \\ 
 Paliacate negro & $\frac{k}{3!}$ & 600 & 800 \\ 
 Paliacate dorado & $\frac{k}{3!}$ & 800 & 1200 \\ 
 Paliacate verde & $\frac{k}{4!}$ & 1200 & 1800 \\
 Paliacate azul & $\frac{15k}{5!}$ & 3000 & 5000 \\ 
 \hline
\end{tabular}
\end{center}


\begin{enumerate}[label=\Alph*)]
\item (2pts) Obtenga el valor de k que hace que f(x) sea una función de distribución.
\item (3pts) Obtenga el valor esperado del costo por botella y la varianza del costo por botella 
\item (3pts) Obtenga el valor esperado del ingreso por botella y la varianza del ingreso por botella 
\item (2pts) Si se venden 1000 botellas al dia, encuentre la utilidad esperada para un día de venta (utilidad=ingreso-costo)
\end{enumerate}

\question \textbf{(10 pts)}
Un alumno de una H. institución decide ir a una fiesta de Halloween aftereco. El número de refrescos (x) que tomará en toda la fiesta sigue la siguiente distribución:


\begin{center}
\begin{tabular}{ |c|c|c|c| } 
\hline
x & p(x) \\
\hline
1 & 0.1 \\
2 & 0.2 \\
3 & 0.25 \\
4 & 0.2 \\
5 & 0.25 \\
\hline
\end{tabular}
\end{center}

\begin{enumerate}[label=\Alph*)]
\item (4pts) Encuentre el valor esperado, la moda, la mediana y la desviación estándar del número de refrescos
\item (4pts) Suponiendo que el aumento en peso (AP) del alumno por refresco consumido sigue la siguiente función $AP(X) = 0.1X^2+0.25X$, encuentre el aumento de peso esperado del alumno
\item (2pts) Obtenga la función de probabilidad acumulada y grafíquela
\end{enumerate}

\question \textbf{(15 pts)}
El consumo mensual de kilos de croquetas de Colmillo sigue la siguiente función de distribución:

\[   
f(x) = 
     \begin{cases}
       Kxe^{-\frac{x}{5}} & x \geq 0\\
       0 & \text{en otro caso} \\
     \end{cases}
\]
$$$$

\begin{enumerate}[label=\Alph*)]
\item (6pts) Determine el valor de k para que f(x) sea una función de distribución
\item (6pts) Obtenga la función de probabilidad acumulada. Compruebe que si X = límite inferior y Y = límite superior, entonces F(X) = 0 y F(Y) = 1
\item (3pts) Si para un mes particular solo se tienen 25 kilos de croqueta, ¿Cuál es la probabilidad que las croquetas sean insuficientes para colmillo?
\end{enumerate}

\question \textbf{(10 pts)}
El monto total de la cuenta en cientos de pesos (x) de un restaurante es una variable alatoria que sigue la siguiente función de densidad:


\[   
f(x) = 
     \begin{cases}
       a+bx & 0 \leq x \leq 10\\
       bx & 10 < x \leq 20\\
       0 & \text{en otro caso} \\
     \end{cases}
\]

Adicional, Se sabe que $P(0 \geq x \geq 10) = 0.5$

\begin{enumerate}[label=\Alph*)]
\item (5pts) Calcule los valores de a y b de tal manera que $f(x)$ sea una función de distribución (Nota, debido a que trabajamos con 4 decimales, puede ser que no integre exactamente 1, pero debe tener un rango de error de $\pm 0.01$)
\item (5pts) La utilidad neta (UN) para el restaurante está en función del monto de la cuenta y sigue la siguiente distribución: 
\end{enumerate}
$$ UN(X) = \frac{X}{2}-c $$
 
Donde $c$ es una constante dada. Obtenga el valor esperado, la moda y la varianza de la utilidad neta. (Usando los resultados de inciso a)

\question \textbf{(10 pts)}

A partir de la siguiente tabla de probabilidad conjunta, calcule las siguientes probabilidades (los valores que toma \textbf{X} se expresan en los renglones, los valores que toma \textbf{Y} se expresan en las columnas):

\begin{center}
\begin{tabular}{ |c|c|c| } 
\hline
p(x,y) & 1 & 2 \\
\hline
0 & 0.1 & 0.2 \\
1 & 0.05 & 0.05 \\
2 & 0.2 & 0.1 \\
3 & 0.1 & 0.2  \\
\hline
\end{tabular}
\end{center}

\begin{enumerate}[label=\Alph*)]
\item Obtenga la distribución marginal de X y la distribución marginal de Y
\item (1pt) Obtenga: P(X<=1) y P(Y>1)
\item (1pt) Obtenga: P(X<=1, y<=1) y P(X<=1, y>1)
\item (1pt) Obtenga: P(Y-X>2)
\item (1pt) Obtenga: P(X>0|Y=2)
\item (4pts) Obtenga: coeficiente de correlación de X y Y
\item (2pts) Obtenga E(Y-X) y Var(Y-X)
\end{enumerate}

\question \textbf{(10 pts)}
Se lanzan dos dados. Sea X el número de unos que salen y Y el número de seis que aparecen.
\begin{enumerate}[label=\Alph*)]
\item (3pts) Obtenga la función de probabilidad conjunta de X y Y
\item (1pts) ¿Son independientes X y Y? Justifique por medio del concepto de independencia de funciones de probabilidad que se vió en clase.  
\item (3pts) Calcule la covarianza y el coeficiente de correlación entre X y Y
\item (3pts) Si Z es el número de dados con valor diferente a 1 y 6, obtenga el valor esperado y la varianza de Z
\end{enumerate}

\question \textbf{(10 pts)}
El periodo de funcionamiento de un iphone hasta su primera falla (en años) se puede modelar por medio de la siguiente función de distribución acumulada:

\[   
F(y) = 
     \begin{cases}
       0 & y<0\\
      1-e^{-y^2} & y \geq 0 \\
     \end{cases}
\]

Es decir la probabilidad de que falle en 0 años o menos es F(0) = 1-exp(0) = 0, la probabilidad que falle en 3 años o menos es F(3) = 1-exp(-9)

\begin{enumerate}[label=\Alph*)]
\item (2pts) Comprueba que F(y) cumple con las propiedades de función de distribución acumulada
\item (2pts) Calcule la proabilidad que el iphone no falle entre 1.5 y 3.5 años
\item (4pts) Calcule la función de densidad correspondiente a la función de distribución acumulada
\item (2pts) Compruebe que f(y) cumple con las propiedades de función de distribución (En caso de no poder integrar f(x), use como argumento la función F(x))
\end{enumerate}


\end{questions} 



\end{document}
  

