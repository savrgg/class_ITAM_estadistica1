%%% Template originaly created by Karol Kozioł (mail@karol-koziol.net) and modified for ShareLaTeX use

\documentclass[addpoints]{exam}

\usepackage[T1]{fontenc}
\usepackage[utf8]{inputenc}
\usepackage{graphicx}
\usepackage{xcolor}

\renewcommand\familydefault{\sfdefault}
\usepackage{tgheros}

\usepackage{amsmath}
\usepackage{amssymb,amsthm,textcomp}
\usepackage{enumerate}
\usepackage{multicol}
\usepackage{tikz}
\usepackage[spanish]{babel}

\usepackage{geometry}
\geometry{left=25mm,right=25mm,bindingoffset=0mm, top=20mm,bottom=20mm}


\linespread{1.3}

\newcommand{\linia}{\rule{\linewidth}{0.5pt}}

% custom theorems if needed
\newtheoremstyle{mytheor}
{1ex}{1ex}{\normalfont}{0pt}{\scshape}{.}{1ex}
{{\thmname{#1 }}{\thmnumber{#2}}{\thmnote{ (#3)}}}
  
  \theoremstyle{mytheor}
  \newtheorem{defi}{Definition}
  
  % my own titles
  \makeatletter
  \renewcommand{\maketitle}{
    \begin{center}
    \vspace{2ex}
    {\huge \textsc{\@title}}
    \vspace{1ex}
    \\
    \linia\\
    \@author \hfill \@date
    \vspace{4ex}
    \end{center}
  }
  \makeatother
  %%%
  
  % custom footers and headers
  %\usepackage{fancyhdr}
  %\pagestyle{fancy}
  \lfoot{Assignment \textnumero{} 5}
  \cfoot{}
  \rfoot{Page \thepage}
  %\renewcommand{\headrulewidth}{0pt}
  %\renewcommand{\footrulewidth}{0pt}
  
  % code listing settings
  \usepackage{listings}
  \lstset{
    language=Python,
    basicstyle=\ttfamily\small,
    aboveskip={1.0\baselineskip},
    belowskip={1.0\baselineskip},
    columns=fixed,
    extendedchars=true,
    breaklines=true,
    tabsize=4,
    prebreak=\raisebox{0ex}[0ex][0ex]{\ensuremath{\hookleftarrow}},
    frame=lines,
    showtabs=false,
    showspaces=false,
    showstringspaces=false,
    keywordstyle=\color[rgb]{0.627,0.126,0.941},
    commentstyle=\color[rgb]{0.133,0.545,0.133},
    stringstyle=\color[rgb]{01,0,0},
    numbers=left,
    numberstyle=\small,
    stepnumber=1,
    numbersep=10pt,
    captionpos=t,
    escapeinside={\%*}{*)}
  }
  
  %%%----------%%%----------%%%----------%%%----------%%%
  
  \begin{document}
  
  \title{Propuesta Final - Estadística I}
  
  \author{ITAM, Primavera 2022}
  
  \date{21/05/2022}
  
  \maketitle
  
  \section*{Instrucciones}
  
\vspace{10pt}
  

  
  \section*{Poisson y geométrica}
  
  \begin{questions}
  
  \question (5 pts) El Centro de Aprendizaje Redacción y Lenguas estudió el número de errores ortográficos que cometen los estudiantes del ITAM al escribir un reporte de una cuartilla (X). Encontraron que la media de errores por cuartilla es de 3. 
  \begin{enumerate}
  \item ¿Qué distribución se puede emplear para modelar los errores (X)?. Determine su media y su varianza\\
  \textcolor{red}{$X \sim Pois(3), \\ E(X) = 3, \\ V(X) = 3$}
  \item Determine ¿Cuál es la probabilidad que un estudiante cometa al menos 4 errores en la cuartilla? ¿Cuál es la probabilidad que un estudiante cometa exactamente 4 errores en la cuartilla? \\
  \textcolor{red}{$P(X >= 4) =  1-P(X <= 3) = 1- 0.647 = 0.353$} \\
  \textcolor{red}{$P(X = 4) =  P(X <= 4)-P(X <= 3) = 0.815 - 0.647 = 0.168$}
  \item La calificación de la cuartilla se puede modelar como la función $C(X)=100-\frac{X^2}{6}$. ¿Cuál es la calificación esperada de esta cuartilla? \\
  \textcolor{red}{$E(X^2) = V(X)+E(X)^2 = 3 + 9 = 12$} \\
  \textcolor{red}{$E(C) = E(100-\frac{X^2}{6}) = 100-\frac{E(X^2)}{6} = 100 - \frac{12}{6} = 98$}

  \item Se ha revisado una proporción de la cuartilla y se sabe que hay al menos 2 errores. ¿Cuál es la probabilidad que se tengan al menos 3 errores en la cuartilla completa? \\
  \textcolor{red}{$P(X >= 3 | X >= 2 ) = \frac{P(X >= 3,X >= 2 )}{P(X >= 2)} = \frac{P(X >= 3)}{P(X>=2)} = \frac{1-P(X<=2)}{1-P(X<=1)} = \frac{0.577}{0.801} = 0.7203$}
  
  \end{enumerate}
  
    \question (5 pts) El número de llamadas que recibe el conmutador del ITAM por hora se puede modelar como una variable Poisson (X) con media de 15. Debido a condiciones del conmutador se conoce que a lo máximo atienda 20 llamadas por hora y deja las demás sin atender. 

  \begin{enumerate}
  \item ¿Cuál es la probabilidad que se dejen llamadas sin atender? \\
  \textcolor{red}{$P(X > 20) = 1- P(X<= 20) = 1-0.917 = 0.083$}
  \item El ITAM incurre en un costo por llamada (tanto por las atendidas como por las no atendidas). El costo se puede modelar como: $C(X) = 100+10X$. Encuentre el Valor Esperado y la varianza del costo. \\
  \textcolor{red}{$E(C) = E(100+10X) = 100+10E(X) = 100+10(15) = 250$} \\
  \textcolor{red}{$V(C) = V(100+10X) = 100V(X) = 100(15) = 150$}
  
  \item Se sabe que se han recibido al menos 10 llamadas en una hora, ¿Cual es la probilidad de que se tengan más de 20 llamadas en esa hora? \\
  \textcolor{red}{$P(X>20|X>=10) = \frac{P(X>20, X>=10)}{P(X>=10)} = \frac{P(X>20)}{P(X>=10)} = \frac{1 - P(X<=20)}{1-P(X<=9)} = \frac{1-0.917}{1-0.07} = \frac{0.083}{0.93} = 0.0892$}

  \item Basándose en el inciso anterior, cual es el valor esperado 

  \end{enumerate}
  
  
  
  
  
  \
  
\end{questions}
  
  \section*{Seccion B: Preguntas a desarrollar (65 pts)}
  
  \begin{questions} 

  \question (10 pts)
  Sean A y B dos eventos tal que $P(A) = 0.5$ y $P(B) = 0.3$. Calcule las siguientes probababilidades: $P(A \cap B)$, $P(A \cup B)$, $P(A|B)$ y $P(A \cup B^c)$ si:
  \begin{enumerate}
  \item A y B son eventos mutuamente excluyentes
  \item A y B son eventos independientes
  \end{enumerate}
  
  \question  (15 pts)
  Una persona compra 10 boletos de una rifa. En total hay 60 boletos y hay 10 boletos con premios iguales.
  \begin{enumerate}
  \item ¿Cuál es la probabilidad de que esta persona gane un premio?
  \item ¿Cuál es la probabilidad de que esta persona gane al menos un premio?
  \end{enumerate}
  
  \question (15 pts) Dos contratos de construcción son asignados de manera aleatoria a tres compañías, A, B y C. Cada compañía puede manejar ninguno, uno o los dos contratos y cada contrato es asignado a una compañia solamente. ¿Cuál es la probabilidad de que:
  \begin{enumerate}
  \item Todos los contratos vayan a compañias diferentes
  \item Entre la compañia A y B tengan todos los contratos, es decir que la compañia C no tenga contratos?
  \end{enumerate}


  \question (15 pts) Se lanzan 3 monedas que no son honestas, es decir sea $P(A)$ la probabilidad de aguila, la moneda A tiene una $P(A) = 2/3$, la moneda B una $P(A) = 1/2$ y la moneda C una $P(A) = 2/5$.
  
  \begin{enumerate}
  \item (5 pts) Escriba el espacio muestral del experimento aleatorio
  \item (5 pts) Obtenga la probabilidad de cada resultado del espacio muestral
  \item (5 pts) ¿Cuál es la probabilidad de observar al menos dos \textbf{soles}?
  \item (5 pts) ¿Cuál es la probabilidad que el último lanzamiento sea águila?
  \end{enumerate}
  
  \question  (10 pts) Un mecánico desea realizar pruebas con un nuevo sistema de frenos para automóviles. El menciona que la probabilidad de falla es de 0.09, mientras que su jefe  tiene la idea de que la probabilidad de falla es de 0.02. Los dos están de acuerdo en que la probabilidad de que exista un choque con otro auto dado que fallaron los frenos es de 0.65, por otra parte, si no fallan, la probabilidad de un choque con otro auto es de solo 0.05. Se realiza una prueba y el auto cocha con otro automoviol:
  \begin{enumerate}
  \item De acuerdo con el mecánico, ¿Cuál es la probabilidad de que haya fallado los frenos dado el accidente?
  \item De acuerdo con el mecánico, ¿Cuál es la probabilidad de que haya fallado los frenos dado el accidente?
  \end{enumerate}

\end{questions}


\section*{Seccion C: Análisis Exploratorio de Datos (15 pts)}

\textbf{Resuelve una de las siguientes dos preguntas:}

\begin{questions}

\question (15 pts) En una fábrica de textiles se aplicó una prueba para medir como aumentaban el número de productos dañados de acuerdo al tiempo promedio que tardaban en producirse. Se cree que entre más rápido realicen los textiles, más productos dañados tendrán. Para esto se tomó la \textbf{muestra} de 15 dias de producción con los respectivos tiempos promedio y el número de productos dañados:

\begin{table}[h]
\centering
\begin{tabular}{llllllllllllllll}
Dia & 1 & 2 & 3 & 4 & 5 & 6 & 7 & 8 & 9 & 10 & 11 & 12 & 13 & 14 & 15 \\ \hline
Tiempo & 25 & 20 & 40 & 28 & 39 & 38 & 10 & 30 & 50 & 22 & 27 & 33 & 34 & 40 & 45 \\
Productos Dañados & 4 & 5 & 2 & 4 & 3 & 3 & 10 & 3 & 2 & 5 & 4 & 3 & 3 & 2 & 2
\end{tabular}
\end{table}

\begin{enumerate}
  \item (5 pts) Construya un diagrama de dispersión con la información de la tabla
  \item (5 pts) ¿Qué concluye del diagrama?
  \item (5 pts) Proponga una medida que indique si existe una asociación lineal entre el tiempo y el número de productos dañados
  \end{enumerate}

 \question (15 pts) Se tienen los siguientes datos que representan los ingresos semanales (X) de una \textbf{muestra} de 15 personas en una empresa manufacturera de automóviles: 1000, 1200, 1300, 1500, 1500, 1600, 1700, 1700, 1700, 1800, 2000, 2200, 2200, 2300, 3300. Adicional, se tiene que $\sum{X_i} = 27000$ y $\sum{X_i^2} = 52,960,000$
 

  \begin{enumerate}
  \item (5 pts) Construya un diagrama de tallo y hoja, donde los tallos indiquen miles y las hojas cientos. 
  \item (5 pts) Calcule la media, mediana, varianza y coeficiente de variación del ingreso semanal proporcionado
  \item (5 pts) Construya un diagrama de caja y brazos (boxplot) para los datos observados
  \end{enumerate}
   
\end{questions} 
\end{document}
  

