%%% Template originaly created by Karol Kozioł (mail@karol-koziol.net) and modified for ShareLaTeX use

\documentclass[addpoints]{exam}

\usepackage[T1]{fontenc}
\usepackage[utf8]{inputenc}
\usepackage{graphicx}
\usepackage{xcolor}

\renewcommand\familydefault{\sfdefault}
\usepackage{tgheros}

\usepackage{amsmath}
\usepackage{amssymb,amsthm,textcomp}
\usepackage{enumerate}
\usepackage{multicol}
\usepackage{tikz}
\usepackage[spanish]{babel}

\usepackage{geometry}
\geometry{left=25mm,right=25mm,bindingoffset=0mm, top=20mm,bottom=20mm}


\linespread{1.3}

\newcommand{\linia}{\rule{\linewidth}{0.5pt}}

% custom theorems if needed
\newtheoremstyle{mytheor}
{1ex}{1ex}{\normalfont}{0pt}{\scshape}{.}{1ex}
{{\thmname{#1 }}{\thmnumber{#2}}{\thmnote{ (#3)}}}
  
  \theoremstyle{mytheor}
  \newtheorem{defi}{Definition}
  
  % my own titles
  \makeatletter
  \renewcommand{\maketitle}{
    \begin{center}
    \vspace{2ex}
    {\huge \textsc{\@title}}
    \vspace{1ex}
    \\
    \linia\\
    \@author \hfill \@date
    \vspace{4ex}
    \end{center}
  }
  \makeatother
  %%%
  
  % custom footers and headers
  %\usepackage{fancyhdr}
  %\pagestyle{fancy}
  \lfoot{Assignment \textnumero{} 5}
  \cfoot{}
  \rfoot{Page \thepage}
  %\renewcommand{\headrulewidth}{0pt}
  %\renewcommand{\footrulewidth}{0pt}
  
  % code listing settings
  \usepackage{listings}
  \lstset{
    language=Python,
    basicstyle=\ttfamily\small,
    aboveskip={1.0\baselineskip},
    belowskip={1.0\baselineskip},
    columns=fixed,
    extendedchars=true,
    breaklines=true,
    tabsize=4,
    prebreak=\raisebox{0ex}[0ex][0ex]{\ensuremath{\hookleftarrow}},
    frame=lines,
    showtabs=false,
    showspaces=false,
    showstringspaces=false,
    keywordstyle=\color[rgb]{0.627,0.126,0.941},
    commentstyle=\color[rgb]{0.133,0.545,0.133},
    stringstyle=\color[rgb]{01,0,0},
    numbers=left,
    numberstyle=\small,
    stepnumber=1,
    numbersep=10pt,
    captionpos=t,
    escapeinside={\%*}{*)}
  }
  
  %%%----------%%%----------%%%----------%%%----------%%%
  
  \begin{document}
  
  \title{Parcial 1 - Estadística II}
  
  \author{ITAM, Primavera 2020}
  
  \date{26/02/2020}
  
  \maketitle
  
  \section*{Instrucciones}
  
El examen consta de dos secciones. La primera es opción múltiple y V/F donde se deberá seleccionar la opción correcta. Para cada pregunta hay una sola respuesta correcta. Si se selecciona más de una opción será considerada como incorrecta. En la segunda sección se deberá desarrollar el problema planteado. Se debe cuidar la formalidad al escribir los resultados, ya que es parte de la calificación del problema. En caso de no tener el desarrollo de la pregunta, o bien se llegué a la respuesta sin una justificación se podrá anular la respuesta. 
  
  \vspace{10pt}
  
El examen tiene una duración de 1:45 horas. \textbf{Cualquier práctica fraudulenta es sancionada de acuerdo al reglamento de la coordinación y el departamento de estadística.} 
  
  \section*{Seccion A: Opción múltiple y V/F(20 pts)}
  
  \begin{questions}
  
  \question (5 pts) CONCEPTOS DE ESTADÍSTICA: Se realizará un estudio donde los elementos de interés son las agencias que venden automóbiles. Identifique si la variable de interés es cualitativa o cuantitativa, discreta o continua y la escala de medición correspondiente. 
  
  \begin{enumerate}
  \item El número de automóviles por agencia
  \item Número de visitas a la agencia por mes
  \item Duración de las visitas más largas por mes
  \item Costo mensual de la renta del local de la agencia
  \item Temperatura en ºC promedio dentro de cada agencia
  \end{enumerate}
  
  \question (9 pts) Determine si las siguientes afirmaciones son verdaderas o falsas. Justifique
  
  \begin{enumerate}
  \item La media de un conjunto de datos no considera el valor máximo y minimo de los datos para su cálculo
  \item La moda puede tener más de tres valores 
  \item Es imposible que la moda sea igual a la mediana y a la media sean iguales
  \item El coeficiente de variación es considerado un parámetro de centralidad
  \item El IQR es considerado un parámetro de dispersión
  \item El primer cuartil es menor o igual al mínimo de los datos
  \item La mediana debe ser igual al menos a uno de los valores observados
  \item La moda debe ser igual al menos a uno de los valores observados
  \item La media debe ser igual al menos a uno de los valores observados
  \end{enumerate}
  
  \question (6 pts) 
  Determine el \textbf{tamaño} del espacio muestral (cuantos elementos tiene, sin enumerarlos) para cada uno de los experimentos:
  \begin{enumerate}
  \item Se lanzan 10 veces una moneda de dos caras
  \item Una urna contiene cinco bolas rojas, seis blancas, una azul y tres amarilla. Una segunda contiene cinco pelotas blancas y tres azules. Una bola es seleccionada de cada urna. 
  \item Las urnas del ejercicio anterior son mezcladas en una urna y se extraen dos de ellas con reemplazo
  \end{enumerate}
  
\end{questions}
  
  \section*{Seccion B: Preguntas a desarrollar (65 pts)}
  
  \begin{questions} 

  \question (10 pts)
  Sean A y B dos eventos tal que $P(A) = 0.5$ y $P(B) = 0.3$. Calcule las siguientes probababilidades: $P(A \cap B)$, $P(A \cup B)$, $P(A|B)$ y $P(A \cup B^c)$ si:
  \begin{enumerate}
  \item A y B son eventos mutuamente excluyentes
  \item A y B son eventos independientes
  \end{enumerate}
  
  \question  (15 pts)
  Una persona compra 10 boletos de una rifa. En total hay 60 boletos y hay 10 boletos con premios iguales.
  \begin{enumerate}
  \item ¿Cuál es la probabilidad de que esta persona gane un premio?
  \item ¿Cuál es la probabilidad de que esta persona gane al menos un premio?
  \end{enumerate}
  
  \question (15 pts) Dos contratos de construcción son asignados de manera aleatoria a tres compañías, A, B y C. Cada compañía puede manejar ninguno, uno o los dos contratos y cada contrato es asignado a una compañia solamente. ¿Cuál es la probabilidad de que:
  \begin{enumerate}
  \item Todos los contratos vayan a compañias diferentes
  \item Entre la compañia A y B tengan todos los contratos, es decir que la compañia C no tenga contratos?
  \end{enumerate}


  \question (15 pts) Se lanzan 3 monedas que no son honestas, es decir sea $P(A)$ la probabilidad de aguila, la moneda A tiene una $P(A) = 2/3$, la moneda B una $P(A) = 1/2$ y la moneda C una $P(A) = 2/5$.
  
  \begin{enumerate}
  \item (5 pts) Escriba el espacio muestral del experimento aleatorio
  \item (5 pts) Obtenga la probabilidad de cada resultado del espacio muestral
  \item (5 pts) ¿Cuál es la probabilidad de observar al menos dos \textbf{soles}?
  \item (5 pts) ¿Cuál es la probabilidad que el último lanzamiento sea águila?
  \end{enumerate}
  
  \question  (10 pts) Un mecánico desea realizar pruebas con un nuevo sistema de frenos para automóviles. El menciona que la probabilidad de falla es de 0.09, mientras que su jefe  tiene la idea de que la probabilidad de falla es de 0.02. Los dos están de acuerdo en que la probabilidad de que exista un choque con otro auto dado que fallaron los frenos es de 0.65, por otra parte, si no fallan, la probabilidad de un choque con otro auto es de solo 0.05. Se realiza una prueba y el auto cocha con otro automoviol:
  \begin{enumerate}
  \item De acuerdo con el mecánico, ¿Cuál es la probabilidad de que haya fallado los frenos dado el accidente?
  \item De acuerdo con el mecánico, ¿Cuál es la probabilidad de que haya fallado los frenos dado el accidente?
  \end{enumerate}

\end{questions}


\section*{Seccion C: Análisis Exploratorio de Datos (15 pts)}

\textbf{Resuelve una de las siguientes dos preguntas:}

\begin{questions}

\question (15 pts) En una fábrica de textiles se aplicó una prueba para medir como aumentaban el número de productos dañados de acuerdo al tiempo promedio que tardaban en producirse. Se cree que entre más rápido realicen los textiles, más productos dañados tendrán. Para esto se tomó la \textbf{muestra} de 15 dias de producción con los respectivos tiempos promedio y el número de productos dañados:

\begin{table}[h]
\centering
\begin{tabular}{llllllllllllllll}
Dia & 1 & 2 & 3 & 4 & 5 & 6 & 7 & 8 & 9 & 10 & 11 & 12 & 13 & 14 & 15 \\ \hline
Tiempo & 25 & 20 & 40 & 28 & 39 & 38 & 10 & 30 & 50 & 22 & 27 & 33 & 34 & 40 & 45 \\
Productos Dañados & 4 & 5 & 2 & 4 & 3 & 3 & 10 & 3 & 2 & 5 & 4 & 3 & 3 & 2 & 2
\end{tabular}
\end{table}

\begin{enumerate}
  \item (5 pts) Construya un diagrama de dispersión con la información de la tabla
  \item (5 pts) ¿Qué concluye del diagrama?
  \item (5 pts) Proponga una medida que indique si existe una asociación lineal entre el tiempo y el número de productos dañados
  \end{enumerate}

 \question (15 pts) Se tienen los siguientes datos que representan los ingresos semanales (X) de una \textbf{muestra} de 15 personas en una empresa manufacturera de automóviles: 1000, 1200, 1300, 1500, 1500, 1600, 1700, 1700, 1700, 1800, 2000, 2200, 2200, 2300, 3300. Adicional, se tiene que $\sum{X_i} = 27000$ y $\sum{X_i^2} = 52,960,000$
 

  \begin{enumerate}
  \item (5 pts) Construya un diagrama de tallo y hoja, donde los tallos indiquen miles y las hojas cientos. 
  \item (5 pts) Calcule la media, mediana, varianza y coeficiente de variación del ingreso semanal proporcionado
  \item (5 pts) Construya un diagrama de caja y brazos (boxplot) para los datos observados
  \end{enumerate}
   
\end{questions} 
\end{document}
  

