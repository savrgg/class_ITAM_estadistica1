\documentclass{../oxmathproblems}
\usepackage{blindtext}
\usepackage{hyperref}
\usepackage{geometry}

\course{ITAM - Estadística 1}
\oxfordterm{Assignment 05}
\sheetnumber{1}
\sheettitle{}
\extrawidth{2cm}
\begin{document}

\begin{questions}

\miquestion\textbf {Distribuciones de probabilidad.}

\text {Tenemos que:}
$$ \text {n = 10 } 
\text {p = .04 }
\text { q = .96 = (1-p)} 
$$
\text {Entonces:} 
P(x $\leq$ 1) = 0.9418

\miquestion\textbf{Distribuciones Poisson.}
\text {Tenemos que:} 

 \text{ y $\sim$ Poisson(7,7)} 
 \text{ Dado que: }
$$ \lambda = 7. $$
 \text{ Entonces: } 
 
 \begin{itemize}
\item  a) 
$$ 
 = P( Y \leq 4) = P( \frac{y- \mu_y}{\sqrt(\sigma^2)} \leq \frac{4-7}{\sqrt(7)}) 
 = 0.1729
$$
\item  b) P(3$<$ Y$\leq$8)
 \text{ Es lo mismo que : } 
$$ P(Y \leq 8) - P(Y < 3)
=  P( \frac{y- \mu_y}{\sqrt(\sigma^2)} \leq \frac{8-7}{\sqrt(7)}) -  P( \frac{y- \mu_y}{\sqrt(\sigma^2)} < \frac{3-7}{\sqrt(7)}) = .643
$$

\item  c) P(3$<$Y$<$8)
\text{ Es lo mismo que : } 
$$ P(Y < 8) - P(Y < 3)
=  P( \frac{y- \mu_y}{\sqrt(\sigma^2)} < \frac{8-7}{\sqrt(7)}) -  P( \frac{y- \mu_y}{\sqrt(\sigma^2)} < \frac{3-7}{\sqrt(7)}) \cong .5169
$$ 
\item  d) P(Y=8$\mid$Y$\leq$10)
\text{ Es lo mismo que : }  
$$ \frac{P( Y=8)}{P( Y \leq 10)} =  \frac{P(\frac{y- \mu_y}{\sqrt(\sigma^2)} =  \frac{8-7}{\sqrt(7)})}{P( \frac{y- \mu_y}{\sqrt(\sigma^2)} \leq \frac{10-7}{\sqrt(7)})} = 0.144
$$
\end{itemize}

\miquestion\textbf{Distribuciones Poisson. }
\text{ Sabemos que : } 
$$ \lambda = 50 $$  
\text{ Entonces : } 
\begin{itemize}
\item a) $\mu_x$, $\sigma_x$
$$ \mu_x = 50 \text{ y } \sigma_x = 50 $$
\item  b) P(x$<$35)
\text{ Es lo mismo que : } 
$$ P(x \leq 34) = .0108 $$
\item  c) P(40$<$x$<$60) $\cong$ 0.8217

\item  d) P(x$>$75)
\text{ Por complemento: }
$$ 1- P( x \leq 74) = 1 - .9994 \cong 0.0005 $$ 
\end{itemize}

\miquestion\textbf{Distribuciones Poisson. }

\text{ Sabemos que las dos condiciones para que sea una función de probabilidad son : }
\begin{itemize}
\item a) $$ f_x (x) \geq 0 $$ 
\item  b) $$ \sum f_x (x) = 1 $$
\end{itemize}

\text{ Entonces: }

\begin{itemize}
\item a) $f(0) = K(0^2 + 4) \geq 0 $
$$ 4K \geq 0 $$
 $$ k \geq 0/ 4$$
$$ k \geq 0 $$

\item  b) \text{ Ahora :  }
$$f(0) = K(0^2 + 4) =  4K $$

$$f(1) = K(1^2 + 4) =  5K $$

$$f(2) = K(2^2 + 4) =  8K $$

$$f(3) = K(3^2 + 4) =  13K $$
\text{ Entonces: } 
4K + 5k + 8k + 13k = 1
$$ K = 1/30 $$
\end{itemize}

\miquestion\textbf{Funciones de Distribución Uniforme. }
\begin{itemize}
\item  a) $\mu_x$, $\sigma_x$
$$ \mu = E(x)= \sum_x xp(x)$$ 
\text{ Entonces:  }
 $\mu_x$ = 2 ($\frac{1}{6} $) + 4($\frac{1}{6} $) + 6($\frac{1}{6} $) + ... = 7 

$$ \sigma_x^2 = E[x^2]-E[x]^2 = 60.6667 - (7)^2 = 11.6667 $$ 
 $ E[x^2] = \sum_x^2 xp(x)  = 60.6667 $ 
 $$ = 2^2 (\frac{1}{6}) + 4^2(\frac{1}{6}) + 6^2(\frac{1}{6}) + ... =  60.6667 $$ 
 
\text{ Entonces:  } 
 $$ \sigma_x  = \sqrt11.6667 = 3.4157 $$ 


\item  b) P[x$>$8] = 2/6

\item  c) P[2$<$x$<$10 $\mid$ x$\geq$4]
= $$  \frac{P(2 < x <10)}{P (x \geq 4)} = \frac{3/6}{5/6} = 0.6 $$ 
\end{itemize}

\miquestion\textbf { Función Geometrica Discreta}
\begin{itemize}
\item a)
\text{ Tenemos que:   } 
\text{ X = Número de llamadas a la estación hasta ser atendido}

 \text {Éxito = llamada respondida}  

\text{ Fracaso = llamada no respondida }
$$ p = .06 \text{ y }
(1 - p )= .94 $$ 
\text{ Entonces :   } 
$$ f(x) = (1-p)^{x-1} (p)  = (0.94)^9(0.06) = 0.0564 $$
\item b) 
$ E(x) = \frac{1}{p} $ 
$ = \frac{1}{0.06} = 16.667 $
\end{itemize}

\miquestion\textbf { Función Geometrica Discreta}

$$ p = .08 \text{ y }
(1 - p )= .92 $$ 

\text{ Entonces :   } 
$$ f(x) = (1-p)^{x-1} (p)  = (0.92)^4(0.08) = 0.0573 $$

\end {questions}

\end{document}