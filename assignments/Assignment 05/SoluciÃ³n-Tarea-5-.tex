\documentclass{../oxmathproblems}
\usepackage{blindtext}
\usepackage{hyperref}
\usepackage{geometry}

\course{ITAM - Estadística 1}
\oxfordterm{Assignment 05}
\sheetnumber{1}
\sheettitle{}
\extrawidth{2cm}
\begin{document}

\begin{questions}

\miquestion\textbf {Distribuciones de probabilidad.}

\text {Tenemos que:}
$$ \text {n = 10 } 
\text {p = .04 }
\text { q = .96 = (1-p)} 
$$
\text {Entonces:} 
P(x $\leq$ 1) = 0.9418



\miquestion\textbf{Distribuciones Poisson. }
\text{ Sabemos que : } 
$$ \lambda = 50 $$  
\text{ Entonces : } 
\begin{itemize}
\item a) $\mu_x$, $\sigma_x$
$$ \mu_x = 50 \text{ y } \sigma_x = \sqrt{50} $$

\item  b) P(x$<$35)
\text{ Es lo mismo que : } 
$$ 
P(x \leq 34)  = $$  
 \text{ al estandarizar : } 
$$ P( \frac{x - \mu_x}{\sigma_x}  \leq \frac{34-50}{\sqrt{50}})  
$$ 
 $ = P(z \leq - 0.32) $  = .3744  % CORREGIDO 
 
 
 
\item  c) P(40$<$x$<$60) 
\text{ Entonces: } 
$  P (40 < x < 60) = P(x < 60) - P(x < 40) = $  
$$ P(\frac{x - \mu_x}{\sigma_x} < \frac{60-50}{\sqrt{50}}) - P(\frac{x - \mu_x}{\sigma_x} < \frac{40-50}{\sqrt{50}}) $$

=$$ P(z < 1.41) - P(z < - 1.41) \cong 0.8414 $$  % CORREGIDO  


\item  d) P(x$>$75)

\text{ Por complemento: }
$$ 1- P( x \leq 74) = 1- P(\frac{x - \mu_x}{\sigma_x}) \leq \frac{74-50}{\sqrt{50}})$$

$$= 1- P(z \leq 3.39) $$
$$= 1 - 0.9997  = 0.0003$$  % CORREGIDO 
\end{itemize}

\miquestion 

\text{ Sabemos que las dos condiciones para que sea una función de probabilidad son : }
\begin{itemize}
\item a) $$ f_x (x) \geq 0 $$ 
\item  b) $$ \sum f_x (x) = 1 $$
\end{itemize}

\text{ Entonces: }

\begin{itemize}
\item a) $f(0) = K(0^2 + 4) \geq 0 $
$$ 4K \geq 0 $$
 $$ k \geq 0/ 4$$
$$ k \geq 0 $$

\item  b) \text{ Ahora :  }
$$f(0) = K(0^2 + 4) =  4K $$

$$f(1) = K(1^2 + 4) =  5K $$

$$f(2) = K(2^2 + 4) =  8K $$

$$f(3) = K(3^2 + 4) =  13K $$
\text{ Entonces: } 
4K + 5k + 8k + 13k = 1
$$ K = 1/30 $$
\end{itemize}

\miquestion\textbf{Funciones de Distribución Uniforme. }
\begin{itemize}
\item  a) $\mu_x$, $\sigma_x$
$$ \mu = E(x)= \sum_x xp(x)$$ 
\text{ Entonces:  }
 $\mu_x$ = 2 ($\frac{1}{6} $) + 4($\frac{1}{6} $) + 6($\frac{1}{6} $) + ... = 7 

$$ \sigma_x^2 = E[x^2]-E[x]^2 = 60.6667 - (7)^2 = 11.6667 $$ 
 $ E[x^2] = \sum_{x^2} xp(x)  = 60.6667 $ 
 $$ = 2^2 (\frac{1}{6}) + 4^2(\frac{1}{6}) + 6^2(\frac{1}{6}) + ... =  60.6667 $$ 
 
\text{ Entonces:  } 
 $$ \sigma_x  = \sqrt11.6667 = 3.4157 $$ 


\item  b) P[x$>$8] = 2/6

\item  c) P[2$<$x$<$10 $\mid$ x$\geq$4]
= $$  \frac{P(2 < x <10)}{P (x \geq 4)} = \frac{3/6}{5/6} = 0.6 $$ 
\end{itemize}

\miquestion\textbf { Función Geometrica Discreta}
\begin{itemize}
\item a)
\text{ Tenemos que:   } 
\text{ X = Número de llamadas a la estación hasta ser atendido}

 \text {Éxito = llamada respondida}  

\text{ Fracaso = llamada no respondida }
$$ p = .06 \text{ y }
(1 - p )= .94 $$ 
\text{ Entonces :   } 
$$ f(x) = (1-p)^{x-1} (p)  = (0.94)^9(0.06) = 0.0564 $$
\item b) 
$ E(x) = \frac{1}{p} $ 
$ = \frac{1}{0.06} = 16.667 $
\end{itemize}

\miquestion\textbf { Función Geometrica Discreta}

$$ p = .08 \text{ y }
(1 - p )= .92 $$ 

\text{ Entonces :   } 
$$ f(x) = (1-p)^{x-1} (p)  = (0.92)^4(0.08) = 0.0573 $$

\end {questions}

\end{document}