\documentclass{../oxmathproblems}
\usepackage{blindtext}
\usepackage{hyperref}
\usepackage{geometry}

\course{ITAM - Estadística 1}
\oxfordterm{Assignment 05}
\sheetnumber{1}
\sheettitle{}
\extrawidth{2cm}
\begin{document}

\begin{questions}


\miquestion \textbf {Distribuciones de probabilidad.} Si un producto tiene defectos de un 4 por ciento de los casos. ¿Cuál es la probabilidad de que en una muestra de tamaño 10, haya a lo más 1 defectuoso?

\miquestion \textbf {Distribución Poisson} La variable aleatoria Y tiene una distribución Poisson con parámetro $\lambda$=7. 
Calcular: 
\begin{itemize}
\item  a) P(Y $\leq$ 4)
\item  b) P(3$<$ Y$\leq$8)
\item  c) P(3$<$Y$<$8)
\item  d) P(Y=8$\mid$Y$\leq$10)
\end{itemize}

\miquestion \textbf {Distribución Poisson} Considere el comportamiento de una variable aleatoria x se puede describir de forma aceptable con una distribución de Poisson con paramétro igual a 50. 
Calcular: 
\begin{itemize}
\item a) $\mu_x$, $\sigma_x$
\item  b) P(x$<$35)
\item  c) P(40$<$x$<$60)
\item  d) P(x$>$75)
\end{itemize}

\miquestion \textbf{Funciones de probabilidad}.  Encuentre el valor de K que hace que la siguiente función sea función de probabilidad. 
 $$f(x) = K(x^2 + 4)$$

	
\miquestion \textbf{Funciones de distribución uniforme} Sea X una variable aleatoria con distribución uniforme en el conjunto {2,4,6,8,10,12}
Calcular: 
\begin{itemize}
\item  a) $\mu_x$, $\sigma_x$
\item  b) P[x$>$8]
\item  c) P[2$<$x$<$10 $\mid$ x$\geq$4]
\end{itemize}
	
\miquestion \textbf {Función Geométrica discreta} Suponga que cada una de sus llamadas a una estación de radio popular tiene una probabilidad de 0.06 de ser respondida. Asumiendo que las  llamadas son independientes, ¿cuál es la probabilidad de que le responsan a la décima llamada? ¿Cuál es el número medio de llamadas para conectar?

\miquestion \textbf {Función Geométrica discreta} La probabilidad de que cierto examen médico dé lugar a una reacción "positiva" es igual a 0.8 ¿Cuál es la probabilidad de que ocurran 5 reacciones "negativas" antes de la primera positiva?



\end{questions}

\end{document}



