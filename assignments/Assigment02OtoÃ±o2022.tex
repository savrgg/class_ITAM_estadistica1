

%Example of use of oxmathproblems latex class for problem sheets
%(un)comment this line to enable/disable output of any solutions in the
% file
%printanswers

\documentclass{oxmathproblems}
\usepackage{blindtext}
\usepackage{hyperref}
\usepackage{geometry}
%define the page header/title info
\course{ITAM - Estadistica 1}
\oxfordterm{Assignment 02}
\sheetnumber{1}
\sheettitle{}

\begin{document}

\begin{questions}

\miquestion \textbf{Variable aleatoria continua.}
El responsable de control de calidad de Mr.Lucky debe verificar el pedo de las bolsas de 2.27 Kg. de cereal. Para cumplir con su tarea sin verificar cada bolsa que sale de la planta de Mr.Lucky, el responsable toma una muestra diariamente de algunas bolsas, pesa el contenido y extrae una conclusión sobre el peso promedio de las bolsas que salen de la planta ese día. 
En la siguiente tabla se muestra el peso de 15 bolsas seleccionadas como muestra en un día, que fueron tomadas de la planta. 
\begin{center}
\begin{tabular}{ |c|c| } 
 \hline
 \textbf{Observaciones} & \textbf{Peso en Kg} \\ 
 \hline
 1 & 2.25 \\
 2  & 2.27 \\
 3 & 2.27\\ 
 4 & 2.27 \\ 
 5 & 2.27 \\ 
 6 & 2.27 \\ 
 7 & 2.27\\ 
 8 & 2.27 \\ 
 9 & 2.36 \\ 
 10 & 2.40 \\ 
11 & 2.40 \\ 
12 & 2.49 \\
13 & 2.50 \\ 
14 & 2.56 \\ 
15 & 2.71 \\ 
 \hline
\end{tabular}
\end{center}

Calcular: 

\begin{itemize}
\item  a) Tabla de frecuencia absoluta y relativa 
\item  b) Medidas de tendencia central ( media, mediana y moda) y de dispersión (varianza, desviación estándar,coeficiente de variación) para datos agrupados
\item  c) Medidas de tendencia central ( media, mediana y moda) y de dispersión (varianza, desviación estándar,coeficiente de variación) para datos no agrupados. 
\item d) Gráfica el histograma, poligono de frecuencias, ojiva, y  diagrama de caja y brazos. 
\end{itemize}


\miquestion \textbf{Coeficiente de correlación y covarianza. }
Se obtienen las medidas de la superficie del área de descanso "x" (en pies cuadrados),y el precio de venta "y", de 12 residencias.

\begin{center}
\begin{tabular}{ |c|c| } 
 \hline
 \textbf{x(pies cuadrados)}  & \textbf{y(en miles)}\\ 
 \hline
 1360 & 278.5 \\
  1940 & 375.7 \\
 1750 & 339.5\\ 
 1550  & 329.8\\ 
 1790 & 295.6\\ 
 1750 & 310.3\\ 
 2230 & 460.5\\ 
 1600 & 305.2\\ 
 1450 & 288.6\\ 
 1870 & 365.7\\ 
 2210 & 425.3\\ 
 1480 & 268.8\\
 \hline
\end{tabular}
\end{center}

Encuentre el coeficiente de correlación y covarianza para el número de pies cuadrados de área de descanso (x) y el precio de venta de una casa (y). 


\end{questions}

\textbf{Bibliografía}
Mendenhall, W. (2006). Introducción a la probabilidad y Estadística (Vol. 13). Cengage Learning.
Aguirre, V. A. B. A. (2006). Fundamentos de Probabilidad y Estadística (2 ed.). Jit Press.



\end{document}
