

\documentclass{oxmathproblems}
\usepackage{blindtext}
\usepackage{hyperref}
\usepackage{geometry}

\course{ITAM - Estadística 1}
\oxfordterm{Assignment 07}
\sheetnumber{1}
\sheettitle{}

\begin{document}

\begin{questions}

\miquestion \textbf {Distribución Uniforme continua.} El tiempo de reacción en segundos, ante un estimulo visual, se considera que es una variable aleatoria que toma valores en los términos [5,9]. Además, cualesquiera dos intervalos de la misma longitud contenidos en [5,9] tienen asociada la misma probabilidad. 
Determine: 
\begin{itemize}
\item  a) ¿El tiempo de reacción es una variable aleatoria o continua? ¿Cuál es la distribución de esta variable aleatoria?
\item  b)Determina la probabilidad de que el tiempo de reacción de una persona sea menor a 7 segundos, si se sabe que ya han pasado más de 6 segundos. 
\item  c) Valor esperado y desviación estándar del tiempo de reacción
\end{itemize}

\miquestion \textbf {Distribución Uniforme continua.} La variación (x) de la cantidad de agua (en cientos de litros) en un depósito de una semana a la siguiente, sigue una distribución uniforme con función de densidad: 

% Para agregar funciones de probabilidad
\[   
f(x) = 
     \begin{cases}
       K & (-20 \leq x \leq 20)\\
       0 & \text{en otro caso} \\
     \end{cases}
\]
$$$$


Determine: 
\begin{itemize}
\item  a) El valor de k 
\item  b) Obtener la función de distribución acumulada de X 
\item  c) ¿Cuál es la probabilidad de que en un mes se tengan dos semanas con valores positivos de x?
\end{itemize}


\miquestion En cierta zona de la ciudad la concentración de monóxido de carbono (w), en un período de dos horas, tiene aproximadamente una distribución exponencial con varianza igual a 6.5 partes por millón. 
Determine: 
\begin{itemize}
\item a) La probabilidad de que la concentración de monóxido exceda las 6 partes por millón
\item b) En un momento dado se cuenta con la información de que w está entre 3 y 9. ¿Cuál es la probabilidad de que w sea mayor a 6?
\end{itemize}





\end{questions}

\end{document}














