%Example of use of oxmathproblems latex class for problem sheets
\documentclass{oxmathproblems}
\usepackage{blindtext}
\usepackage{hyperref}
%(un)comment this line to enable/disable output of any solutions in the file
%\printanswers

%define the page header/title info
\course{ITAM - Estadistica 1}
\oxfordterm{Assignment 01}
\sheetnumber{1}
\sheettitle{}

\begin{document}

\begin{questions}

\miquestion \textbf{Describa el proceso de inferencia estadística}

\miquestion \textbf{¿Por qué es importante la probabilidad en el proceso de inferencia estadística?}

\miquestion \textbf{¿Cómo se clasifican las variables de acuerdo a la escala de medición? Proporcione un ejemplo de cada caso}

\miquestion \textbf{En una encuesta sobre percepción de seguridad en las familias se consideran las siguientes variables:}
\begin{itemize}
\item Ingreso mensual de la familia
\item Número de integrantes en la familia
\item Alcaldía donde vive la familia
\item Tipo de población donde viven (urbana/rural)
\item Tipo de transporte que usan
\end{itemize}
Indica para cada variable si es cualitativa o cuantitativa, discreta o continua y su escala de medición.

\miquestion \textbf{Una agencia para protección del medio ambiente realiza pruebas sobre el rendimiento de la gasolina en los automóviles nuevos. En una prueba la agencia reportó que después de seleccionar 20 automóviles nuevos de un modelo en particular y realizar algunas pruebas, se obtuvo un rendimiento promedio de 14 km por litro:}
\begin{itemize}
\item a) Especifique la población de interes y la muestra
\item b) Describa la variable estudiada, ¿Qué tipo de datos son utilizados y en que escala de medición?
\item c) ¿Considera esta muestra una muestra representativa?
\item d) ¿Cuál es la población estadística?
\end{itemize}

\end{questions}

\end{document}