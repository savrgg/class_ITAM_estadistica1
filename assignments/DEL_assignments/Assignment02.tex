%Example of use of oxmathproblems latex class for problem sheets
\documentclass{oxmathproblems}
\usepackage{blindtext}
\usepackage{hyperref}
%(un)comment this line to enable/disable output of any solutions in the file
%\printanswers

%define the page header/title info
\course{ITAM - Estadistica 1}
\oxfordterm{Assignment 02}
\sheetnumber{1}
\sheettitle{}

\begin{document}

\begin{questions}

\miquestion Se realiza una muestra de 10 elementos del precio de renta de un departamento cerca del ITAM. Los datos se muestran a continuación:

\{20000, 25000, 18000, 15000, 15000, 40000, 19000, 20000, 25000, 20000\}

Construya un diagrama de puntos con estos datos.

\miquestion Se enumeran los ingresos brutos mensuales de 10 exalumnos del ITAM, los cuales se listan a continuación:

$\{30000, 40000, 120000, 150000, 200000, 50000, 40000, 80000, 90000, 70000\}$

Construya un diagrama de tallo y hojas con estos datos. Usted proponga la escala del tallo y las hojas.

\miquestion Se tiene los siguientes datos de la variable X: número de materias cursadas por 20 estudiantes del itam:

$\{3,4,7,4,3,5,7,5,3,3,5,4,4,5,6,6,7,7,5,5\}$

\begin{itemize}
\item a) Construya una tabla de frecuencia absoluta y relativa con estos datos.
\item b) Calcule la media, la mediana, clase modal (moda) ¿Los datos presentados son datos agrupados o no agrupados?
\end{itemize}

\miquestion De acuerdo a una escala Richter, se han medido los 50 temblores más recientes en la Ciudad de México. Las distribución de frecuencias es la siguiente:
\begin{center}
\begin{tabular}{ |c|c| } 
 \hline
 \textbf{Clases} & \textbf{$n_i$} \\ 
 \hline
 2.25-2.75 & 4 \\
 2.75-3.25 & 2 \\
 3.25-3.75 & 5\\ 
 3.75-4.25 & 8 \\ 
 4.25-4.75 & 12 \\ 
 4.75-5.25 & 8 \\ 
 5.25-5.75 & 5 \\ 
 5.75-6.25 & 3 \\ 
 6.25-6.75 & 2 \\ 
 6.75-7.25 & 1 \\ 
 \hline
\end{tabular}
\end{center}

\begin{itemize}
\item a) Calcule para cada intervalo de clase su frecuencia relativa, su frecuencia absoluta acumulada y su frecuencia acumulada relativa
\item b) Obtenga el polígono de frecuencias
\item c) Calcule la media, la mediana, clase modal (moda) ¿Los datos presentados son datos agrupados o no agrupados?
\end{itemize}

\end{questions}

\end{document}