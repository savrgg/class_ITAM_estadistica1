\documentclass{oxmathproblems}
\usepackage{blindtext}
\usepackage{hyperref}
\usepackage{geometry}

\course{ITAM - Estadística 1}
\oxfordterm{Assignment 03}
\sheetnumber{1}
\sheettitle{}

\begin{document}

\begin{questions}

\miquestion \textbf{Teoría de conjuntos y Probabilidad} Ejercicio 3.1.49. Se carga un dado de manera que los números impares tienen el triple de probabilidad de aparecer que  los números pares. Si se lanza el dado de una vez, ¿cuál es la probabilidad de que: 
\begin{itemize}
\item  a) el número sea par?
\item  b) el número sea primo?
\item  c) el número sea impar y mayor que dos?
\end{itemize}

\miquestion Ejercicio 3.1.53. El gerente de una tienda desea emplear a 3 personas de entre 4 solicitantes de sexo masculino y 6 de sexo femenino. Obtener la probabilidad de emplear 
\begin{itemize}
\item  a) solamente a un hombre
\item  b) como máximo a dos hombres
\item  c) a dos hombres y una  mujer
\end{itemize}

\miquestion Considere el experimento de lanzar 2 dados. Considere así mismo los siguientes eventos: Sea 
A={La suma de los números obtenidos al lanzar los dados es impar}
B={El número obtenido en el primer dado es 1}
C={La suma es 7}
\begin{itemize}
 \item a) ¿Son A y B independientes?
\item b) ¿Son A y C independientes?
\item c) ¿Son B y C independientes?
\end{itemize}
\miquestion \textbf 
Ejercicio 3.1.26. Los eventos A y B son tales que P(B)=1/4, y P(A)=1/2. Obtenga P[A|B] e indique si los eventos tienen alguna relación si: ||
\begin{itemize}
 \item a) P[B|A] = 1/4
 \item b) P[B|A] = 1/2
 \item  c) P[B|A] = 1/3
\end{itemize}

\miquestion Sea P[AuB]=0.68 y P[A]=0.25. Determine P[B] si: 
\begin{itemize}
 \item a) A y B son mutuamente excluyentes 
 \item b) A y B son independientes
 \item  c) P[B|A] = 0.35 
\end{itemize}


\end{questions}






\end{document}
