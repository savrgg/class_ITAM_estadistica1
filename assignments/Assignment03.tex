%Example of use of oxmathproblems latex class for problem sheets
\documentclass{oxmathproblems}
\usepackage{blindtext}
\usepackage{hyperref}
%(un)comment this line to enable/disable output of any solutions in the file
%\printanswers

%define the page header/title info
\course{ITAM - Estadistica 1}
\oxfordterm{Assignment 02}
\sheetnumber{1}
\sheettitle{}

\begin{document}

\begin{questions}

\miquestion \textbf{Datos Cualitativos y distribuciones de frecuencia}

\textbf{Distribución de frecuencia}
\textbf{Diagrama circular}
\textbf{Diagrama de barras}

\miquestion \textbf{Datos Cuantitativos y distribuciones de frecuencia}

\textbf{Diagrama de punto}

Construya un diagrama de puntos con estos datos.

\textbf{Diagrama de tallo y hojas}
Se enumeran los ingresos brutos mensuales de 10 exalumnos del ITAM, los cuales se listan a continuación:

$\{30000, 40000, 120000, 150000, 200000, 50000, 40000, 80000, 90000, 70000\}$

Construya un diagrama de tallo y hojas con estos datos. Usted proponga la escala del tallo y las hojas.


\textbf{Diagrama de frecuencias variables discretas}
Se tiene los siguientes datos de la variable X: número de materias cursadas por 20 estudiantes del itam:

$\{3,4,7,4,3,5,7,5,3,3,5,4,4,5,6,6,7,7,5,5\}$

Construya un diagrama de frecuencias con estos datos.

\textbf{Diagrama de frecuencias variables continuas}
\textbf{Histograma}
\textbf{Polígono de frecuencias}



\begin{itemize}
\item a) Especifique la población de interes y la muestra
\item b) Describa la variable estudiada, ¿Qué tipo de datos son utilizados y en que escala de medición?
\item c) ¿Considera esta muestra una muestra representativa?
\item d) ¿Cuál es la población estadística?
\end{itemize}

\end{questions}

\end{document}