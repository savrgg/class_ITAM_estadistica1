\documentclass{../oxmathproblems}
\usepackage{blindtext}
\usepackage{hyperref}
\usepackage{geometry}

\course{ITAM - Estadística 1}
\oxfordterm{Assignment 06}
\sheetnumber{1}
\sheettitle{}
\extrawidth{2cm}
\begin{document}

\begin{questions}

\miquestion\textbf{Función de densidad}
\[   
f(x) = 
     \begin{cases}
       \frac{2000}{(x+100)^3} & x > 0 \\
       0 & \text{en otro caso} \\
     \end{cases}
\]
$$$$
Determinar: 
\begin{itemize}
\item  a) La probabilidad de que un paquete de ese medicamento tenga una vida útil de a lo más 200 días. 

\text{Entonces: }

 \begin{equation}
\int_{0}^{200} \frac{2000}{(x+100)^3} \cdot dx
\end{equation}

= \begin{equation}
(2000 )\int_{0}^{200} (x+100)^{-3} \cdot dx
\end{equation}
= 2000[$\frac{(x+100)^{-2}}{-2}]^{200}_{0}$  = 
2000[$\frac{1}{-2(300)^2}]$ = 
2000[$\frac{1}{-180 000}]$


\item  b) La probabilidad de entre 90 y 130 días.

\text{Entonces: }

 \begin{equation}
\int_{90}^{130} \frac{2000}{(x+100)^3} \cdot dx
\end{equation}

= \begin{equation}
(2000 )\int_{90}^{130} (x+100)^{-3} \cdot dx
\end{equation}
= 2000[$\frac{(x+100)^{-2}}{-2}]^{130}_{90}$  = 
2000[$\frac{1}{-2(130 + 100)^2}]$[$\frac{1}{-2(90 + 100)^2}]$ = 
2000(0.0000043) $\cong$ 0.0087

\item  c) La vida útil promedio del medicamento. 
\text{Tenemos que calcular E(X), entonces:  }

 \begin{equation}
\int x(\frac{2000}{(x+100)^3}) \cdot dx
\end{equation}

=  \begin{equation}
\int \frac{2000x}{(x+100)^3} \cdot dx
\end{equation}

=   \begin{equation}
\int 2000 (\frac{x}{(x+100)^3}) \cdot dx
\end{equation}

= 
$2000 (\frac{x^2}{2})(\frac{(x+100)^{-2}}{-2})$

= $2000 (\frac{x^2(x+100)^{-2}}{4})$ =  $2000 (\frac{x^2}{4(x+100)^{2}})$ = 0.8737


\item  d) La varianza de la vida útil del medicamento: 
\text{ Sabemos que } 
$$ \sigma_x^2 = E[x^2]-E[x]^2$$ 
\text{Entonces } 
\begin{equation}
\int x^2(\frac{2000}{(x+100)^3}) \cdot dx
\end{equation}

=  \begin{equation}
\int \frac{2000x^2}{(x+100)^3} \cdot dx
\end{equation}

=   \begin{equation}
\int 2000 (\frac{x^2}{(x+100)^3}) \cdot dx
\end{equation}

= \begin{equation}
2000 \int  (\frac{x^2}{(x+100)^3}) \cdot dx
\end{equation}
= 39.44

\text{ Ahora:  }
$$ \sigma_x^2 = E[x^2]-E[x]^2$$  = $39.44 - (0.8737)^2 = 38.67$
\end{itemize}

\miquestion\textbf{Función de densidad}
\[ 
f(x) = 
     \begin{cases}
       \frac{c}{\sqrt{x}} & 0 <x < 4 \\
       0 & \text{en otro caso} \\
     \end{cases}
\]
$$$$
Encuentre:  
\begin{itemize}
\item  a) El valor de c

\begin{equation}
\int_{0}^{4} \frac{c}{\sqrt{x}} \cdot dx
\end{equation}
= \begin{equation}
\int_{0}^{4} \frac{c}{x^{1/2}} \cdot dx
\end{equation}
= \begin{equation}
\int_{0}^{4} c(x^{1/2}) \cdot dx
\end{equation}

= c[$\frac{(x^{1/2}}{1/2}]^{4}_{0}$  = 1

= $ 4c = 1 $ 

$ c = \frac{1}{4} $ 

\item  b) Función de distribución acumulada. 

\begin{equation}
\int_{0}^{4} \frac{1/4}{\sqrt{x}} \cdot dx
\end{equation}
= 
\begin{equation}
\int_{0}^{4} \frac{1}{4\sqrt{x}} \cdot dx
\end{equation}
= 
\begin{equation}
 \frac{1}{4} \int_{0}^{4} \frac{1}{\sqrt{x}} \cdot dx
\end{equation}
= 
$$ \frac{1}{4}[\frac{x^{1/2}}{1/2}]^{4}_{0}
$$
= $$ \frac{1}{4}[2x^{1/2}]^{4}
$$
= $\frac{1}{2}\sqrt{x}$


\item  c) La probabilidad de que x menor a 1/4
\begin{equation}
\int_{-\infty}^{1/4} \frac{1}{4\sqrt{x}} \cdot dx
\end{equation} 
=  $[\frac{1}{2}\sqrt{x}]_{-\infty}^{1/4}$ = $\frac{1}{2}\sqrt{1/4}$
= $\frac{1}{2}$ 


\item  d) La probabilidad de que x mayor a 1
\begin{equation}
 \int_{1}^{\infty} \frac{1}{4\sqrt{x}} \cdot dx
\end{equation}

=  $[\frac{1}{2}\sqrt{x}]_{1}^{\infty}$ = - $[\frac{1}{2}\sqrt{1}]$ = - $\frac{1}{4}$ 

\end{itemize}

\miquestion\textbf{Función de densidad}
\begin{itemize}
\item  a) 
 P(40 $<$ X $<$ 50) = 
\begin{equation}
 \int_{40}^{50} \frac{x-30}{450} \cdot dx
\end{equation} 
= $$ \frac{1}{450} [\frac{x^2}{2}-30x]^{50}_{40} = \frac{1}{450}(150) = \frac{1}{3} $$ 

\item  b) La demanda promedio de bolsas 

\begin{equation}
 x \int_{40}^{50} \frac{x-30}{450} \cdot dx
\end{equation} 
= $$ \frac{1}{450} [\frac{x^3}{3}-30x]^{50}_{40} $$ = 44.51


\item  c) La varianza de la demanda de bolsas

$$ \sigma_x^2 = E[x^2]-E[x]^2 $$ 

\text{Entonces $E[x^2]$ } 
\begin{equation}
x^2 \int_{40}^{50} \frac{x-30}{450} \cdot dx
\end{equation} 
= $$ \frac{1}{450} [\frac{x^4}{4}-30x]^{50}_{40} $$
\end{itemize} 
= 2049.33

\text{Entonces $\sigma_x^2$ } 
= $ 2049.33 - 44.51^2 = 67.43$ 


\miquestion \textbf{Función de densidad} Sea x las ventas mensuales de chocolates, está dada por la siguiente función de densidad: 
\[ 
f(x) = 
     \begin{cases}
       x- \frac{x^3}{4}  & 0  \leq x  \leq 2 \\
       0 & \text{en otro caso} \\
     \end{cases}
\]
$$$$
\begin{itemize}
\item  a) Encontrar la función de distribución f(x) 
\begin{equation}
 \int_{0}^{2} x- \frac{x^3}{4} \cdot dx
\end{equation} 


\item  b) Calcular la varianza

$$ \sigma_x^2 = E[x^2]-E[x]^2 $$ 

\text{Primero: $E[x]$}
= \begin{equation}
 x \int_{0}^{2} x- \frac{x^3}{4} \cdot dx
\end{equation}
= 16/15

\text{Entonces $E[x^2]$ } 

\begin{equation}
 x^2 \int_{0}^{2} x- \frac{x^3}{4} \cdot dx
\end{equation} 
= 4/3


\text{Entonces $\sigma_x^2$ } 
=$  4/3 - (16/15)^2 = 0.1955 $
\end{itemize}

\miquestion \textbf {Bivariado Continuo. } Sea (X,Y) una variable aleatoria bivariada con función de densidad: 
\[ 
f(x) = 
     \begin{cases}
        K  & 0  < {y}   
        {x} < 1 \\
       0 & \text{en otro caso} \\
     \end{cases}
\]


\begin{itemize}
\item  a) Encontrar el valor de K tal que la función sea función de densidad 

\begin{equation}
z = \int _{-\infty}^{\infty} \int _{-\infty}^{\infty} f(x,y) \cdot dx \cdot dy
\end{equation} 
= 1 
= 
 \begin{equation}
 \int _{0}^{1} \int _{0}^{x} k \cdot dx \cdot dy
\end{equation}


 \begin{equation}
 \int _{0}^{1} K (\int _{0}^{x}  \cdot dy) \cdot dx
\end{equation} 

= \begin{equation}
 \int _{0}^{1} K [y]^x_0 \cdot dx
\end{equation} 

= 
\begin{equation}
K \int _{0}^{1}  x \cdot dx
\end{equation}
$
= K [\frac{x^2}{2}]^1_0 
= \frac{K}{2} = 1
$
\text {Entonces:} 
K = 2

\text {Por lo tanto:} 
\[ 
f(x) = 
     \begin{cases}
        2  & 0  < {y}   
        {x} < 1 \\
       0 & \text{en otro caso} \\
     \end{cases}
\]

\item  b) Encontrar las funciones de densidad marginales. ¿Son X e Y independientes?

\text {Primero: } 

\begin{equation}
f_1(x)= \int _{-\infty}^{\infty} f(x,y)  \cdot dy
\end{equation} 
 = 
\begin{equation}
\int _{0}^{x} 2  \cdot dy
\end{equation} 
= $2[y]^x_0 $
= 2x    \text { para : }  0 $<$ x $<$ 1 


\text {Ahora } 

\begin{equation}
f_2(y)= \int _{-\infty}^{\infty} f(x,y)  \cdot dx
\end{equation} 
 = 
\begin{equation}
\int _{0}^{x} 2  \cdot dx
\end{equation} 
= $2[x]^1_y $
= 2- 2y    \text { para : }  0 $<$ y $<$ 1 

\text { X e Y son independientes cuando  }
$f(x,y)= f_1(x)*f_2(y) $

$f_1(x)*f_2(y) = 2x*(2-2y) = 4x-4xy \neq 2 = f(x,y)$

\item  c) Encontrar las funciones de distribución marginales. 

\begin{equation}
F_1(x) = \int _{-\infty}^{X} f_1(t)  \cdot dt
\end{equation} 
= 
\begin{equation}
 \int _{0}^{X} f_1(t)  \cdot dt
\end{equation} 
= 
\begin{equation}
 \int _{0}^{X} 2t  \cdot dt
\end{equation} 
= $ [t^2]_0^x = x^2 $  \text { para : }  0 $<$ x $<$ 1 



\begin{equation}
F_2(y) = \int _{-\infty}^{y} f_2(t)  \cdot dt
\end{equation} 
= 
\begin{equation}
 \int _{0}^{y} f_2(t)  \cdot dt
\end{equation} 
= 
\begin{equation}
 \int _{0}^{y} (2-2t)  \cdot dt
\end{equation} 
= $ [2t - t^2]_0^y = 2y - y^2 $  \text { para : }  0 $<$ y $<$ 1 


\item  d) Encontrar las funciones de densidad 

$f(x/y) = \frac{f(x,y)}{f_2(y)} = \frac{2}{2-2y}$ \text { para : }  0 $<$ y $<$ 1      0 $<$ x $<$ 1



$f(y/x) = \frac{f(x,y)}{f_1(x)} = \frac{2}{2x} $ \text { para : }  0 $<$ x $<$ 1      0 $<$ y $<$ 1
\end{itemize}

\end{questions}



\textbf{Bibliografía}
Aguirre, V. A. B. A. (2006). Fundamentos de Probabilidad y Estadística (2 ed.). Jit Press.


\end{document}


