\documentclass{../oxmathproblems}
\usepackage{blindtext}
\usepackage{hyperref}
\usepackage{geometry}

\course{ITAM - Estadística 1}
\oxfordterm{ Respuestas Assignment 03}
\sheetnumber{1}
\sheettitle{}
\extrawidth{2cm}
\begin{document}

\begin{questions}

\miquestion \textbf{Teoría de conjuntos y Probabilidad}

\text {Sabemos que: }

$$ P(par) = t $$

$$ P(impar) = 3t $$

Hay 3 números pares $(2,4,6)$, entonces sabemos que $P(X = 2)+ P(X = 4) + P(X = 6) = t$, entonces $P(X = 2) = P(X = 4) = P(X = 6) = t/3$. La misma logica lleva que $P(X = 1) = P(X = 3) = P(X = 5) = t$

¿Cuanto vale t? Sabemos que $P(par)+P(impar) = 1$, entonces $t = 1/4$
\text {Entonces: }

\begin{itemize}

\item a) El número sea par: 1/4
\item b) El número sea primo:
\text {Es la suma de la probabilidad de {1,2,3,5} : } 10/12  
\item c) El número sea impar y mayor que dos {3,5}: 2/4 = 1/2

\end{itemize}

\miquestion
\begin{itemize}

\item a) P(solo un hombre)= $\frac{\binom{4}{1}\binom{6}{2}}{\binom{10}{3}}$ = 0.5
\item b) P(como maximo a dos hombres)= $\frac{\binom{4}{0}\binom{6}{3}}{\binom{10}{3}}+ \frac{\binom{4}{1}\binom{6}{2}}{\binom{10}{3}} + \frac{\binom{4}{2}\binom{6}{1}}{\binom{10}{3}} = 0.1666 + 0.5 + 0.3 = 0.9666$
\item c) P(Dos hombres y una mujer) = $\frac{\binom{4}{2}\binom{6}{1}}{\binom{10}{3}}  = 0.3$
\end{itemize}

\miquestion

\begin{itemize}


\item a) Se sabe que $P(A \mid B) = \frac{P(B \mid A) P (A) }{P(B)}$ Entonces: 
$P(A \mid B) =  \frac{P(B \mid A) P (A) }{P(B)} = \frac{\frac{1}{4} \frac{1}{2} }{\frac{1}{4}} = 1/2 = P(A)$. Entonces implica independencia

\item b) $P(A \mid B) = \frac{P(B \mid A) P (A) }{P(B)} = \frac{\frac{1}{2} \frac{1}{2} }{\frac{1}{4}} = 1$, Entonces implica que conociendo a B, tenemos completa certidumbre de A. Esto sucede si A es subconjunto de B

\item c) $P(B \cap A) = P(B \mid  A) P(A) = 0$, Entonces implica que son mutuamente excluyentes

\end{itemize}


\miquestion 

\begin{itemize}

\item a) 
Al ser mutuamente excluyentes implica que: $P(A \cap B) = 0$
\text {Entonces:} 
$$P(A \cup B)= P(A) + P(B) - P(A\cap B) = .43$$

\item b)
Al ser independientes implica que : $P(A \cap B) = P(A)*P(B) $
\text {Entonces:}
$$P(A \cup B)= P(A) + P(B) - P(A\cap B)$$
$$0.68 = 0.25 + P(B) - 0.25*P(B) $$
$$P(B) = \frac{0.43}{0.75} = 0.57$$

\item c) al saber que $P(B|A) = 0.35$ entonces se sabe que $P(B \cap A) = P(B \mid A)P(A)$ Entonces: 

$$ P(A \cup B)= P(A) + P(B) - P(A \cap B) $$
$$ P(A \cup B) = P(A) + P(B) - P(B \mid A) P(A)$$ 
$$ 0.68 = 0.25 + P(B) - 0.35 * 0.25$$ 
$$ P(B)= 0.5175 $$ 

\end{itemize}


\end{questions}

\end{document}
