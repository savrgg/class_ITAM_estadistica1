\documentclass{../oxmathproblems}
\usepackage{blindtext}
\usepackage{hyperref}
\usepackage{geometry}

\course{ITAM - Estadística 1}
\oxfordterm{ Respuestas Assignment 03}
\sheetnumber{1}
\sheettitle{}
\extrawidth{2cm}
\begin{document}

\begin{questions}

\miquestion \textbf{Teoría de conjuntos y Probabilidad}

\text {Sabemos que: }

$ P(par) = t $

$ P(impar) = 3t $
\text {Entonces: }

$ x = 1/4 $

\begin{itemize}

\item a) El número sea par: 1/4

\item b) El número sea primo:
\text {Es la suma de la probabilidad de {1,2,3,5} : } 10/12  

\item c) El número sea impar y mayor que dos: 1/2

\end{itemize}



\miquestion
\begin{itemize}

\item a) P(solo un hombre)= 1/10 

\item b) 2/10
\item c) 3/10
\end{itemize}


\miquestion 
\begin{itemize}

\item a) Sí, P(A $\cap$ B) es diferente de cero 
\item b) C es subconjunto de A 
\item c) Sí, P(B $\cap$ C) es diferente de cero 
\end{itemize}

\miquestion

\begin{itemize}

\item a) P(A$\mid$B)=1/2 = P(A), independientes 
\item b) P(A$\mid$B) = 1 

\item c) P(A$\mid$B) = 2/3  
\end{itemize}

\miquestion 

\begin{itemize}

\item a) 
P(A$\cap$B) = 0 

\text {Entonces:} 

P(A $\cup$ B)= P(A) + P(B) - P(A$\cap$B) = .43
\item b)
P(A $\cap$ B) = P(A)*P(B) 

\text {Entonces:}
P(A $\cup$ B)= P(A) + P(B) - P(A$\cap$B)
.68 = .25 + P(B) - P(A)P(B) 

$P(B) = .43/.75 = .57$
\item c)
P(A $\cup$ B)= P(A) + P(B) - P(A$\cap$B) 

.68 = .25 + P(B) - P(A)P(B$\mid$A) 

$ P(B)= .5175$ 

\end{itemize}


\end{questions}

\end{document}
