\documentclass{oxmathproblems}
\usepackage{blindtext}
\usepackage{hyperref}
\usepackage{geometry}

\course{ITAM - Estadística 1}
\oxfordterm{Assignment 06}
\sheetnumber{1}
\sheettitle{}

\begin{document}

\begin{questions}

\miquestion La vida útil de un medicamento es una variable aleatoria con la siguiente función de densidad: 
\[   
f(x) = 
     \begin{cases}
       \frac{2000}{(x+100)^3} & x > 0 \\
       0 & \text{en otro caso} \\
     \end{cases}
\]
$$$$
Determinar: 
\begin{itemize}
\item  a) La probabilidad de que un paquete de ese medicamento tenga una vida útil de a lo más 200 días. 
\item  b) La probabilidad de entre 90 y 130 días.
\item  c) La vida útil promedio del medicamento. 
\item  d) La varianza de la vida útil del medicamento
\end{itemize}

\miquestion La función de densidad de probabilidad de la variable aleatoria x está dada por: 
\[ 
f(x) = 
     \begin{cases}
       \frac{c}{\sqrt{x}} & 0 <x < 4 \\
       0 & \text{en otro caso} \\
     \end{cases}
\]
$$$$
Encuentre:  
\begin{itemize}
\item  a) El valor de c
\item  b) La probabilidad de que x sea menor a 1/4
\item  c) La probabilidad de que x mayor a 1
\item  d) La función de distribución acumulada y calcule nuevamete b) y c)
\end{itemize}

\miquestion Suponga que un frabicante mexicano de bolsas ha decidido exportar a Estados Unidos. Un despacho de consultoría estadística ha encontrado que la demanda X del producto, expresada en miles de pesos, esta dada por la siguiente función de densidad: 
\[ 
f(x) = 
     \begin{cases}
       \frac{(X-30)}{450}  & 30  \leq x  \leq 60 \\
       0 & \text{en otro caso} \\
     \end{cases}
\]
$$$$
Calcule: 
\begin{itemize}
\item  a) La probabilidad de que la demanda se encuentre entre 40 y 50 miles de pesos. 
\item  b) La demanda promedio de bolsas 
\item  c) La varianza de la demanda de bolsas
\end{itemize}

\miquestion Sea x las ventas mensuales de chocolates, está dada por la siguiente función de densidad: 
\[ 
f(x) = 
     \begin{cases}
       x- \frac{x^3}{4}  & 0  \leq x  \leq 2 \\
       0 & \text{en otro caso} \\
     \end{cases}
\]
$$$$
\begin{itemize}
\item  a) Encontrar la función de distribución f(x)  
\item  b) Calcular la varianza
\end{itemize}


\end{questions}

\textbf{Bibliografía}
Aguirre, V. A. B. A. (2006). Fundamentos de Probabilidad y Estadística (2 ed.). Jit Press.


\end{document}
