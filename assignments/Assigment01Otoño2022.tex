
%Example of use of oxmathproblems latex class for problem sheets
%(un)comment this line to enable/disable output of any solutions in the file
%\printanswers
\documentclass{oxmathproblems}
\usepackage{blindtext}
\usepackage{hyperref}
\usepackage{geometry}
%define the page header/title info
\course{ITAM - Estadistica 1}
\oxfordterm{Assignment 01}
\sheetnumber{1}
\sheettitle{}

\begin{document}

\begin{questions}

\miquestion \textbf{Datos Cualitativos y distribuciones de frecuencia}

Identifica cada una de las siguientes variables como cualitativas o cuantitativas y su escala de medición. 
\begin{itemize}
\item a) El uso más frecuente de su horno de microondas (recalentar, descongelar, calentar, otros)

\item b) El número de consumidores que se niegan a contestar una encuesta por telefono 

\item c) La puerta escogida por un ratón en un experimento de laboratorio (A,B o C)

\item d) El tiempo ganador para un caballo que corre en el Derby de Kentucky 

\item e) El número de niños en un grupo de quinto grado que leen al nuvel de ese grado o mejor. 
\end{itemize}

\miquestion \textbf{Distribución de frecuencias y diagrama de puntos.}
Se enumeran los colores de 21 dulces: 

{Café,
rojo,
amarillo,
café,
anaranjado,
amarillo,
verde,
rojo,
anaranjado,
azul,azul,
café,verde,
verde,azul,café
,azul,café,azul
,café,anaranjado}


Construya la tabla de distribución de frecuencia para la variable cualitativa y su respectivo diagrama de puntos e histograma

\miquestion \textbf{Diagrama de tallo y hoja variables discretas.}
Se tiene los siguientes precios de zapatos deportivos para el primer trimestre del 2022

$\{90,65,75,70,70,68,70,70,60,68,70,74,65,75,70,40,70,95,65\}$

Construya un diagrama de tallo y hoja para estos datos. Usted proponga la escala del tallo y las hojas.

\miquestion \textbf{Variable aleatoria continua}
El responsable de control de calidad de "Mr. Tuky Hot Dog" debe verificar el peso de las bolsas de 2.27 kg. de salchicha. Para cumplir con su tarea sin verificar cada bolsa que sale de la planta, el responsable muestra diariamente algunas bolsas, pesa el contenido y extrae una conclusión sobre el peso promedio de las bolsas que salen de esa planta ese día. 
Se presenta el peso de 15 bolsas seleccionadas en un día: 

$\{2.15,2.18,2.19,2.19,2.19,2.19,2.21,2.27,2.27,2.27,2.27,2.27,2.27,2.30,2.33\}$

Calcular: 
\begin{itemize}
\item a) La tabla de distribución de frecuencias
\item b) Diagrama de tallo y hoja
\item c) Diagrama de puntos 
\item d) Histograma 
\end{itemize}

\end{questions}

\end{document}
