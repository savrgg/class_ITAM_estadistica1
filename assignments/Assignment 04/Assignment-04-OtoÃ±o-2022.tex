\documentclass{../oxmathproblems}
\usepackage{blindtext}
\usepackage{hyperref}
\usepackage{geometry}

\course{ITAM - Estadística 1}
\oxfordterm{Assignment 04}
\sheetnumber{1}
\sheettitle{}
\extrawidth{2cm}

\begin{document}

\begin{questions}


\miquestion \textbf {Variable aleatorias discretas.}
Identifique cada una de las siguientes variables aleatorias como continuas o discretas. 
\begin{itemize}
\item  a) Litros de gasolina vendidos en un día martes en una gasolinería. 
\item  b) El número de llamadas que entran al conmutador del ITAM entre las 8 y las 10 de la mañana
\item  c) El tiempo que un alumno tarda en resolver la tarea de Estadística I. 
\item  d) El número de robos ocurrido en la CDMX 
\item  e) La distancia a la que se puede elevar un papalote
\end{itemize}

\miquestion Una tienda de electrónica vende un modelo particular de computadora portátil. Hay solo cuatro computadoras en existencia y la gerente se pregunta cuál será la demanda de hoy para ese modelo en particular. Ella se entera en el departamento de marketing de que la distribución de probabilidad para x, la demanda diaria para la laptop, es como se muestra en la tabla.  

\begin{center}
\begin{tabular}{ |c|c| } 
 \hline
 \textbf{x} & \textbf{p(x)} \\ 
 \hline
 0 & .10 \\
 1  & .40 \\
2 & .20\\ 
 3 & .15 \\ 
4 & ? \\ 
5 & .05 \\ 
 \hline
\end{tabular}
\end{center}
 Determine:   
\begin{itemize}
\item  a) la probabilidad de que x=4
\item  b) el valor esperado, varianza y desviación estándar de x
\item  c) ¿Es probable que cinco o más clientes deseen comprar una laptop?
\end{itemize}


\miquestion En una lotería realizada a beneficio de una institución local de caridad, se han de vender 8000 boletos a 10 pesos cada uno. El premio es un automóvil de 24 000. Si usted compra dos boletos. ¿Cuál es su ganancia esperada?

\miquestion En un experimento de preferencia de color, ocho juguetes se ponen en un recipiente. Los juguetes son idénticos excepto por el color, dos son rojos y seis son verdes. Se pide a un niño que escoja dos juguetes al azar. ¿Cuál es la probabilidad de que el niño escoja los dos juguetes rojos?

\miquestion Sea "y" el salario inicial (en miles) por hora y "x" el número de años de experiencia de un profesor de maestría. 
\begin{center}
\begin{tabular}{ |c|c| } 
 \hline
 \textbf{x} & \textbf{y} \\ 
 \hline
 2 & 6 \\
 3  & 7.5 \\
4 & 8\\ 
5 & 12 \\ 
6 & 13 \\ 
7 & 15.5 \\ 
 \hline
\end{tabular}
\end{center}
Calcular   
\begin{itemize}
\item  a) el salario medio esperado por hora
\item  b) el número promedio de años de experiencia 
\item  c) La varianza y desviación estándar para "x" y para "y". 
\end{itemize}

\end{questions}

\textbf{Bibliografía}
Mendenhall, W. (2006). Introducción a la probabilidad y Estadística (Vol. 13). Cengage Learning.

\end{document}
