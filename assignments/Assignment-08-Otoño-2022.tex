\documentclass{oxmathproblems}
\usepackage{blindtext}
\usepackage{hyperref}
\usepackage{geometry}

\course{ITAM - Estadística 1}
\oxfordterm{Assignment 08}
\sheetnumber{1}
\sheettitle{}

\begin{document}

\begin{questions}

\miquestion \textbf {Variable aleatorias y distribuciones conjuntas.} En una comunidad urbana de la periferia del D.F, en México, se investigó el ingreso monetario de las parejas que forman un núcleo familiar. Se construyó una tabla de doble entrada, donde X representa el ingreso del cónyuge masculino (miles de pesos) y Y el ingreso del cónyuge femenino. 
La función de probabilidad conjunta de X y Y es: 

\begin{center}
\begin{tabular}{ |c|c|c|c|c|c|c| } 
\hline
Ingreso del esposo X (en miles)/ Ingreso de la esposa Y (en miles) & 0 & 1 & 2 & 3 & 4 & $\varSigma$ \\
\hline
1 & 0.11 & 0.03 & 0.01 & 0.01 & 0 & 0.16 \\
2 & 0.25 & 0.10 & 0.04 & 0.01 & 0 & 0.40 \\
3 & 0.03 & 0.08 & 0.02 & 0.02 & 0 & 0.15 \\
4 & 0.02 & 0.07 & 0.06 & 0.03 & 0.01 & 0.19  \\
5 & 0.01 & 0.02 & 0.01 & 0.02 & 0.04 & 0.10  \\
$\varSigma$ & 0.42 & 0.30 & 0.14 & 0.09 & 0.05 & 1.00  \\
\hline
\end{tabular}
\end{center}

Determinar la probabilidad de que el esposo tenga un ingreso de 2000 y la esposa no tenga un ingreso. 

\miquestion Se lanza un dado y se consideran las variables aleatorias
 $$ X = \text{"doble del resultado del dado"} $$
 
 \[   
Y = 
     \begin{cases}
       1 &  \text{si el resultados del dado es impar}\\
       2 & \text{si el resultado del dado es impar} \\
     \end{cases}
\]
$$$$
Determinar: 
\begin{itemize}
\item  a) La función de probabilidad conjunta de (X, Y)
\item  b) La probabilidad de Y sea par si X es menor o igual a 8. 
\end{itemize}

\miquestion La demanda de refrescos (X) y papitas (y) tienen una distribución normal bivariada con vector de medias $\begin{pmatrix}5 \\ 10 \end{pmatrix} $  y matriz de varianzas-covarianzas $\begin{pmatrix}4 & 1\\ 1 &9\end{pmatrix} $

Calcular:  
\begin{itemize}
\item  a) ¿Cuál es el coeficiente de correlación entre las demandas del bien X y Y? ¿Son bienes complementarios o sustitutos?
\item  b) ¿Cuál es la probabilidad de que la demanda de refrescos (x) supere a la de papitas (y)?
\end{itemize}

\miquestion \textbf {Aproximación de una Binomial a una normal. } El 25 por ciento de las viviendad de una región tiene conexión a internet. Se eligen 80 viviendas y se pide: 
Calcular:  
\begin{itemize}
\item  a)La probabilidad de que al menos 20 de ellas estén conectadas a internet. 
\item  b) El número esperado de viviendas no conectadas al internet 
\item  c)La probabilidad de que el número de viviendas con internet esté entre 10 y 30. 
\end{itemize}

\end{questions}

\textbf{Bibliografía}
Mendenhall, W. (2006). Introducción a la probabilidad y Estadística (Vol. 13). Cengage Learning.

\end{document}

